\documentclass[a4paper,landscape,8pt,fleqn]{scrartcl}
\usepackage[english]{babel}
\usepackage[utf8]{inputenc}
\usepackage[a4paper, landscape, margin=1.1cm]{geometry}
\usepackage{latexsym}
\usepackage{multicol}
\usepackage{amsmath}
\usepackage{amsfonts}
\usepackage{amssymb}
\usepackage{array}
\usepackage{graphicx}
\usepackage{booktabs}
\usepackage{empheq}			% emphasize (box) equations
\usepackage{float}				% add H as an option for floats
\usepackage{parskip}			% add no indentation to new paragraphs
\usepackage{enumitem}		% description lists
\usepackage{fancyhdr}
\usepackage{lastpage}
\usepackage{framed}
%\usepackage{dutchcal}		% lower case calligraphic letters

\newcommand{\SummaryTitle}{Continuous Time Quantitative Finance}
\newcommand{\SummaryAuthor}{Fabian MARBACH}
\newcommand{\SummarySemester}{Spring Semester 2016}

\pagestyle{plain}
\columnsep 30pt
\columnseprule .4pt

\setlength{\mathindent}{0.1cm}

\newcommand*\widefbox[1]{\fbox{\hspace{2em}#1\hspace{2em}}}		% required for boxing several lines of equations at once

\renewcommand*{\familydefault}{\sfdefault}		% set font to default sans-serif

\renewcommand{\labelitemi}{\tiny$\blacksquare$}		% change symbol of itemized lists
\setlist[itemize]{leftmargin=0.4cm}								% reduce indentation of itemized lists
\renewcommand{\labelenumi}{(\roman{enumi})}			% change counter of enumerated lists
\setlist[enumerate]{leftmargin=0.4cm}							% reduce indentation of enumerated lists

\renewcommand{\arraystretch}{1}
\renewcommand{\emph}[1]{\textbf{#1}}

%\allowdisplaybreaks	% equations can be split on two pages/columns

\graphicspath{{images/}}

\pagestyle{fancy}
\fancyhead{}
\setlength{\headheight}{0pt}
\setlength{\footheight}{14pt}
\renewcommand{\headrulewidth}{0pt}
\renewcommand{\footrulewidth}{0.5pt}
\lfoot{\SummaryAuthor}
\cfoot{p \thepage\ / \pageref{LastPage}}
\rfoot{\SummaryTitle}

\usepackage{parskip}	% add no indentation to new paragraphs

\makeatletter
\renewcommand{\section}{\@startsection{section}{1}{0mm}%
{-2\baselineskip}{0.8\baselineskip}%
{\hrule depth 0.2pt width\columnwidth\hrule depth1.5pt
width0.25\columnwidth\vspace*{1.2em}\Large\bfseries}}
\makeatother

\makeatletter
\renewcommand{\subsection}{\@startsection{subsection}{1}{0mm}%
{-2\baselineskip}{0.8\baselineskip}%
{\hrule depth 0.2pt width\columnwidth\hrule depth0.75pt
width0.25\columnwidth\vspace*{1.2em}\large\bfseries}}
\makeatother

\makeatletter
\renewcommand{\subsubsection}{\@startsection{subsubsection}{1}{0mm}%
{-2\baselineskip}{0.8\baselineskip}%
{\hrule depth 0.2pt width\columnwidth\vspace*{1.2em}\normalsize\bfseries}}
\makeatother

\newcommand{\Mx}[1]{\begin{bmatrix}#1\end{bmatrix}}
\newcommand{\dd}[2]{\frac{\text{d}#1}{\text{d}#2}}
\newcommand{\DD}[2]{\frac{\text{D}#1}{\text{D}#2}}
\newcommand{\deidei}[2]{\frac{\partial#1}{\partial#2}}
\newcommand{\Lbrace}[1]{\left\{\begin{array}{ll}#1\end{array}\right.} % left brace with text

% Declare mathematical operators
\DeclareMathOperator{\erf}{erf}					% error function
\DeclareMathOperator{\Var}{Var}				% variance
\DeclareMathOperator{\Varn}{Varn}			% variation
\DeclareMathOperator{\QV}{QV}				% quadratic variation
\DeclareMathOperator{\Cov}{Cov}				% covariance
\DeclareMathOperator{\Ber}{Ber}				% Bernoulli distribution
\DeclareMathOperator{\Bin}{Bin}				% Binomial distribution
\DeclareMathOperator{\Geom}{Geom}		% Geometric distribution
\DeclareMathOperator{\Poi}{Poi}					% Poisson distribution
\DeclareMathOperator{\Gammaa}{Gamma}	% Gamma distribution
\DeclareMathOperator{\Cau}{Cau}				% Cauchy distribution
\DeclareMathOperator{\Adj}{Adj}				% Adjoint
\DeclareMathOperator{\Bias}{Bias}				% Bias of an estimator
\DeclareMathOperator{\rank}{rank}				% Bias of an estimator

\begin{document}
\part*{Summary: \SummaryTitle}
\SummaryAuthor, \SummarySemester
\begin{multicols*}{4}
%\tableofcontents
%\end{multicols}
%{\vspace*{0.3cm}}
%{\hrule depth 0.2pt}
%{\vspace*{0.3cm}}
%\begin{multicols}{2}
\raggedcolumns
\newpage

\section{Basic Black-Scholes model}

\subsection{Dynamics}

\paragraph{$\mathbb{P}$-dynamics}

\begin{align*}
\frac{dS_t}{S_t} &= \mu dt + \sigma dB_t \\
S_t &= S_0 \cdot \exp \left( \left( \mu - \frac{1}{2} \sigma^2 \right) t + \sigma W_t \right)
\end{align*}
where
\begin{multicols}{2}
\begin{description}[style=multiline,leftmargin=0.5cm]
\item[$r$] risk-free rate
\item[$\mu$] trend
\item[$\sigma$] volatility
\item[$B_t$] $\mathbb{P}$-BM
\end{description}
\end{multicols}

\paragraph{$\mathbb{Q}$-dynamics}

\begin{align*}
\frac{dS_t}{S_t} &= r dt + \sigma dB_t \\
S_t &= S_0 \cdot \exp \left( \left( r - \frac{1}{2} \sigma^2 \right) t + \sigma W_t \right)
\end{align*}
with $W_t$ a $\mathbb{Q}$-BM, $\theta = \frac{\mu-r}{\sigma}$ the risk-premium and $\exists$ an EMM $\mathbb{Q}$ s.t.
\begin{align*}
\left. \mathbb{Q} \right|_{\mathcal{F}_t} &= \exp \left( -\theta B_t - \frac{1}{2} \theta2 t \right) \cdot \left. \mathbb{P} \right|_{\mathcal{F}_t}
\end{align*}

\subsection{BS formula}

\paragraph{BS formula for a European option}
\begin{align*}
C_t &= S_t \Phi (d_1) - K e^{-r(T-t)} \Phi (d_2) \\
P_t &= K e^{-r(T-t)} \Phi (-d_1) - S_t \Phi (-d_2) \\
d_{1,2} &= \frac{\log \frac{S_t}{K} + (r \pm \frac{\sigma^2}{2})(T-t)}{\sigma \sqrt{(T-t)}}
\end{align*}
with $d_2 = d_1 -\sigma \sqrt{T-t}$.

\paragraph{BS PDE}
\begin{align*}
\partial_t C + r s \partial_s C + \frac{1}{2} \sigma^2 s^2 \partial^2_{ss} - r C &= 0 & t \in (0,T) \\
C(s,t=T) &= g(s)
\end{align*}
$\forall s > 0$ with $s$ the stock price in linear space, $T$ the maturity and $g(s)$ the payoff function.

\paragraph{Self-financing portfolio}
\begin{align*}
dV_t &= \alpha_t dC_t + \beta_t dS_t
\end{align*}

\paragraph{Hedging ratio}
\begin{align*}
\frac{\beta_t}{\alpha_t} &= -\partial_s C(S_t,t)
\end{align*}

\paragraph{Martingale approach}
\begin{align*}
&C(S_0,T) \\
&\quad = \mathbb{E}_\mathbb{Q} \left[ e^{-r T} (S_T - K) \mathbb{I}_{S_T \geq K} \right] \\
&\quad = \mathbb{E}_\mathbb{Q} \left[ e^{-r T} S_T \mathbb{I}_{S_T \geq K} \right] - e^{-r T} K \mathbb{Q} \left[ S_T \geq K \right]
\end{align*}

\subsection{Greeks}

\paragraph{Delta}
\begin{align*}
\Delta &= \frac{\partial C}{\partial s} \\
\Delta_E^C &= \Phi(d_1) > 0 \\
\Delta_E^P &= -\Phi(-d_1) = \Phi(d_1) - 1 < 0
\end{align*}

\paragraph{Gamma}
\begin{align*}
\Gamma &= \frac{\partial^2 C}{\partial s^2} & \Gamma_E^C &= \Gamma_E^P = \frac{\phi(d_1)}{s \sigma \sqrt{T-t}} > 0
\end{align*}

\paragraph{Rho}
\begin{align*}
\rho &= \frac{\partial C}{\partial r} \\
\rho_E^C &= K(T-t) e^{-r(T-t)} \Phi(d_2) > 0 \\
\rho_E^P &= -K(T-t) e^{-r(T-t)} \Phi(-d_2) < 0
\end{align*}

\paragraph{Theta}
\begin{align*}
\Theta &= \frac{\partial C}{\partial t} \\
\Theta_E^C &= -\frac{s \varphi(d_1) \sigma}{2\sqrt{T-t}} - r K e^{-r(T-t)} \Phi(d_2) < 0 \\
\Theta_E^P &= -\frac{s \varphi(d_1) \sigma}{2\sqrt{T-t}} + r K e^{-r(T-t)} \Phi(-d_2) > 0
\end{align*}

\paragraph{Vega}
\begin{align*}
\mathcal{V} &= \frac{\partial C}{\partial \sigma} & \mathcal{V}_E^C &= \mathcal{V}_E^P = s \varphi(d_1) \sqrt{T-t} > 0
\end{align*}

\subsection{Mathematical tools}

\paragraph{Girsanov's theorem}
The stochastic exponential or Doléans exponential is defined as:
\begin{align*}
\mathcal{E}(X)_t &= \exp \left( X_t - \frac{1}{2} [X]_t \right)
\end{align*}
Define the change of measure as:
\begin{align*}
\left. \frac{d\mathbb{Q}}{d\mathbb{P}} \right|_{\mathcal{F}_t} &= \mathcal{E}(X)_t
\end{align*}
Then:
\begin{itemize}
\item it holds for the expectation of the RV $\xi$:
\begin{align*}
\mathbb{E}_\mathbb{Q}[\xi] = \mathbb{E}_\mathbb{P}\left[ \xi \cdot \left. \frac{d\mathbb{Q}}{d\mathbb{P}} \right|_{\mathcal{F}_t} \right]
\end{align*}
\item if $W_t$ is a $\mathbb{P}$-BM, then a $\mathbb{Q}$-BM is defined as:
\begin{align*}
W_t^\mathbb{Q} &= W_t^\mathbb{P} - \left[ X,W^\mathbb{P} \right]_t
\end{align*}
(e.g. if $X_t = \lambda W_t^\mathbb{P}$, then $W_t^\mathbb{Q} = W_t^\mathbb{P} - \lambda t$)
\end{itemize}

\paragraph{Laplace transform}
\begin{itemize}
\item In general, the Laplace transform is defined as:
\begin{align*}
\mathcal{L}\lbrace f(t) \rbrace(s) = F(s) = \int_0^\infty e^{-s t} f(t) dt
\end{align*}
\item In probability theory: $X$ is a a RV with PDF $f$. Then:
\begin{align*}
\mathcal{L}\lbrace f \rbrace = \mathbb{E} \left[ e^{-s X} \right]
\end{align*}
Remark: Replace $s$ with $-t$ IOT obtain the MGF (moment-generating function) of $X$, i.e. $\mathbb{E} \left[ e^{t X} \right]$.
\end{itemize}

\paragraph{Leibnitz's rule for differentiation under the integral}
\begin{align*}
&\frac{d}{d\alpha} \int_{a(\alpha)}^{b(\alpha)} f(x,\alpha) dx = \frac{db(\alpha)}{d\alpha} f(b(\alpha),\alpha)\\
&\qquad - \frac{da(\alpha)}{d\alpha} f(a(\alpha),\alpha) + \int_{a(\alpha)}^{b(\alpha)} \frac{\partial f(x,\alpha)}{\partial \alpha} dx
\end{align*}

\section{American Currency Options}

\paragraph{Garman-Kohlhagen model}
Under the risk-neutral probability $\mathbb{Q}$:
\begin{align*}
\frac{dS_t}{S_t} &= (r-\delta) dt + \sigma dW_t
\end{align*}
with:
\begin{multicols}{2}
\begin{description}[style=multiline,leftmargin=0.5cm]
\item[$r$] domestic risk-free rate
\item[$\delta$] foreign risk-free rate
\item[$\sigma$] currency volatility
\item[$W_t$] $\mathbb{Q}$-BM
\end{description}
\end{multicols}

\paragraph{American options}
\begin{align*}
C_A(S_0,T) &= \sup_{\tau \in \mathcal{T}(T)} \mathbb{E}_\mathbb{Q} \left[ e^{-r \tau} (S_\tau - K)^+ \right] \\
P_A(S_0,T) &= \sup_{\tau \in \mathcal{T}(T)} \mathbb{E}_\mathbb{Q} \left[ e^{-r \tau} (K - S_\tau)^+ \right]
\end{align*}

\paragraph{PDE (continuation region)}
In the continuation region, an American option satisfies the same PDE as a European option, i.e.
\begin{align*}
& \frac{1}{2} \sigma^2 x^2 \partial^2_{xx} C_A(x,u) + (r-\delta) x \partial_x C_A(x,u) \\
& \qquad - r C_A(x,u) - \partial_u C_A(x,u) = 0
\end{align*}

\paragraph{Impact of parameters on exercise boundaries}
In case of an American currency call:
\begin{itemize}
\item if the volatility $\sigma$ increases: \\
then the exercise boundary increases (i.e. we wait longer)
\item if the domestic interest rate $r$ increases: \\
then the exercise boundary increases (i.e. we wait longer)
\item if the foreign interest rate $\delta$ increases: \\
then the exercise boundary decreases (i.e. we wait less)
\end{itemize}

\subsection{Decompositions}

\paragraph{American currency call price}
\begin{align*}
& C_A(S_t,T-t) = C_E(S_t,T-t) \\
&+ \delta S_t \int_t^T e^{-\delta (s-t)} \Phi(d_1(S_t,b_c(T-s),s-t)) ds \\
&- r K \int_t^T e^{-r(s-t)} \Phi(d_2(S_t,b_c(T-s),s-t)) ds
\end{align*}
with
\begin{align*}
d_1(x,y,u) &= \frac{\log \frac{x}{y} + \left( r-\delta+\frac{1}{2}\sigma^2 \right) u}{\sigma \sqrt{u}} \\
d_2(x,y,u) &= d_1(x,y,u) - \sigma \sqrt{u}
\end{align*}

\paragraph{American currency put price}
\begin{align*}
& P_A(S_t,T-t) = P_E(S_t,T-t) \\
&+ r K \int_t^T e^{-r(s-t)} \Phi(-d_2(S_t,b_p(T-s),s-t)) ds \\
&- \delta S_t \int_t^T e^{-\delta (s-t)} \Phi(-d_1(S_t,b_p(T-s),s-t)) ds
\end{align*}
with $d_1, d_2$ as defined above.

\subsection{Perpetual American currency options}

Now: $C_A(x) = C_A(x,+\infty)$ and $P_A(x) = P_A(x,+\infty)$

\paragraph{PDE approach}
\begin{align*}
& \frac{1}{2} \sigma^2 x^2 C_A''(x) + (r-\delta) x C_A'(x) - r C_A(x) = 0
\end{align*}
with the following boundary conditions (continuity and smooth-pasting):
\begin{align*}
C_A(L^\ast) = L^\ast - K \qquad C_A'(L^\ast) = 1
\end{align*}

\paragraph{Martingale approach}
\begin{align*}
C_A(S_t) &= \sup_\tau \mathbb{E}_\mathbb{Q} \left[ \left. e^{-r(\tau-t)} (S_\tau - K) \right| \mathcal{F}_t \right]
\end{align*}
By continuity of BM:
\begin{align*}
C_A(S_0) &= \sup_L (L - K) \mathbb{E}_\mathbb{Q} \left[ e^{-r T_L} \right]
\end{align*}
Compute derivative $\partial_L (L-K) \mathbb{E}_\mathbb{Q} \left[ e^{-r T_L} \right] = 0$ IOT get $L^\ast$.

\paragraph{Perpetual American call \& put}
\begin{itemize}
\item Continuation region ($x < L^\ast$)
\begin{align*}
C_A(x) &= (L^\ast - K) \left( \frac{x}{L_1^\ast} \right)^{\gamma_1} \\
P_A(x) &= (K - L^\ast) \left( \frac{x}{L_2^\ast} \right)^{\gamma_2}
\end{align*}
with
\begin{align*}
L_{1/2}^\ast &= \frac{\gamma_{1,2}}{\gamma_{1,2} - 1} K \quad \geq K \\
\gamma_{1,2} &= \frac{-\nu \pm \sqrt{\nu^2 + 2r}}{\sigma} \\
 \nu &= \frac{1}{\sigma} \left( r  -\delta - \frac{1}{2} \sigma^2 \right)
\end{align*}
i.e. $\gamma_{1,2}$ are the positive and negative root of:
\begin{align*}
\frac{1}{2} \sigma^2 \gamma^2 + \left( r - \delta - \frac{1}{2} \sigma^2 \right) \gamma - r = 0
\end{align*}
\item Stopping region ($x \geq L^\ast$)
\begin{align*}
C_A(x) &= x-K \\
P_A(x) &= K-x
\end{align*}
i.e. the option price simply corresponds to its intrinsic value.
\end{itemize}

\paragraph{Put-Call symmetry}
\begin{align*}
P_A(S_0,K,r,\delta) &= C_A(K,S_0,\delta,r)
\end{align*}
which comes from the fact that the right to sell a foreign currency corresponds to the right to buy the domestic one.

\paragraph{Perpetual exercise boundaries}
\begin{align*}
b_c(K,r,\delta,T-t) \cdot b_p(K,\delta,r,T-t) = K^2
\end{align*}

\paragraph{Laplace transforms of hitting times}
\begin{itemize}
\item \emph{Standard $\mathbb{Q}$-BM $W_t$} \\
If $T_y$ is the first hitting time of $y \in \mathbb{R}$ for a standard $\mathbb{Q}$-BM, i.e.
\begin{align*}
T_y = \inf \left \lbrace t \geq 0 : W_t = y \right \rbrace
\end{align*}
then the Laplace transform of $T_y$ is given as:
\begin{align*}
\mathbb{E}_\mathbb{Q} \left[ e^{-\frac{1}{2} \lambda^2 T_y} \right] = e^{-\lambda |y|}
\end{align*}
\item \emph{Drifted BM $W_t + \nu t$} \\
Let $T_y$ be the first hitting time of $y \in \mathbb{R}$ for a drifted BM $W_t + \nu t$, i.e.
\begin{align*}
T_y = \inf \left \lbrace t \geq 0 : W_t + \nu t = y \right \rbrace
\end{align*}
Then the corresponding measure $\mathbb{Q}^\ast$ is given according to Girsanov's theorem by:
\begin{align*}
\left. \frac{\mathbb{Q}^\ast}{\mathbb{Q}} \right|_{\mathcal{F}_t} &= e^{-\frac{1}{2} \nu^2 t - \nu W_t}, \qquad W_t^\ast = W_t + \nu t
\end{align*}
and the Laplace transform of $T_y$ is given as:
\begin{align*}
\mathbb{E}_\mathbb{Q} \left[ e^{-\frac{1}{2} \lambda^2 T_y} \right] = e^{\nu y} e^{-|y| \sqrt{\nu^2 + \lambda^2}}
\end{align*}
\end{itemize}

\paragraph{Reflection principle}
\begin{itemize}
\item If $W_t$ a $\mathbb{Q}$-BM, $\tau_m$ the first passage time of $W_t$ at the level $m$ and if another level $\omega < m$ is considered, then:
\begin{align*}
\mathbb{Q} \left[ \tau_m \leq t, W_t \leq \omega \right] = \mathbb{Q} \left[ W_t \geq 2m - \omega \right]
\end{align*}
\item If $M_t$ is the running maximum of $W_t$, then:
\begin{align*}
\mathbb{Q} \left[ M_t \geq m, W_t \leq \omega \right] = \mathbb{Q} \left[ W_t \geq 2m - \omega \right]
\end{align*}
\end{itemize}

\section{Stochastic Volatility}

\paragraph{Time change}
\begin{align*}
\Sigma_T &= \int_0^T \sigma_u^2 du \\
\int_0^t \sigma_u dB_u &= B^\ast_{\Sigma_t} = B^\ast_{\int_0^t \sigma_u^2 du}
\end{align*}
where $B_t$ is the original $\mathbb{Q}$-BM and $B_t^\ast$ is the time-changed $\mathbb{Q}$-BM. \\
For $B_t^\ast$, it holds that:
\begin{align*}
B_{\Sigma_t}^\ast &\sim \mathcal{N}(0,\Sigma_t), \qquad \left[ B^\ast \right]_t = \int_0^t \sigma_u^2 du
\end{align*}

\paragraph{General $\mathbb{P}$-dynamics for a currency}
\begin{align*}
\frac{dS_t}{S_t} &= \mu dt + \sigma_t d \tilde B_t \\
\frac{d\sigma_t}{\sigma_t} &= f(\sigma_t) dt + \gamma d \tilde W_t
\end{align*}
with $\tilde W_t, \tilde B_t$ two BM with correlation coefficient $\rho$.

\paragraph{General $\mathbb{Q}$-dynamics for a currency}
\begin{align*}
\frac{dS_t}{S_t} &= (r - \delta) dt + \sigma_t dB_t \\
\frac{d\sigma_t}{\sigma_t} &= \left( f(\sigma_t) - \Phi_t^\sigma \right) dt + \gamma d W_t
\end{align*}
with $W_t, B_t$ two BM with correlation coefficient $\rho$ and $\Phi_t^\sigma$ the risk premium associated with volatility.

Under $\mathbb{Q}$, the underlying price $S_T$ is then given by:
\begin{align*}
S_t &= S_0 e^{(r-\delta)t - \frac{1}{2} \int_0^t \sigma_u^2 du + \int_0^t dB_u} \\
&= S_0 e^{(r-\delta)t - \frac{1}{2} \Sigma_t^2 + B^\ast_{\Sigma_t}}
\end{align*}

\subsection{Hull \& White model}

\paragraph{$\mathbb{Q}$-dynamics of the Hull \& White model}
Volatility $\sigma$ is assumed to follow a GBM.
\begin{align*}
\frac{dS_t}{S_t} &= r dt + \sigma_t dW_t^{(1)} \qquad \frac{d\sigma_t}{\sigma_t} = k dt + \gamma dW_t
\end{align*}
where $W_t^{(1)}, W_t$ two independent BM.

Assumptions:
\begin{description}[style=multiline,leftmargin=1cm,font=\textbf]
\item[$\delta = 0$] stock option on domestic market
\item[$\rho = 0$] volatility follows a GBM and is uncorrelated with the stock price
\item[$\Phi_\sigma = 0$] volatility has zero systematic risk
\end{description}
Thus, the drift of volatility is assumed to be constant, i.e. $f(\sigma) = k$.

Additional random variable: $V_T = \Sigma_T / T$.

\paragraph{Hull \& White PDE}
Apply Itô's lemma to $C_E = f(S_t,\sigma_t)$ under $\mathbb{Q}$ and use the martingale property $\mathbb{E}_\mathbb{Q}[C_E] = r C_E$ IOT obtain:
\begin{align*}
&\frac{1}{2} \sigma^2 x^2 \partial^2_{xx} C_E + \frac{1}{2} \gamma^2 \sigma^2 \partial^2_{\sigma \sigma} C_E + r x \partial_x C_E \\
&\qquad + \sigma k \partial_\sigma C_E + \partial_t C_E - r C_E = 0
\end{align*}

\subsection{Scott model}

\paragraph{$\mathbb{Q}$-dynamics of the Scott model}
The Scott model assumes that the logarithm of the volatility follows a Vasicek process, i.e. $f(\sigma) = \beta (a - \log \sigma) + \frac{1}{2} \gamma^2$. Then:
\begin{align*}
\frac{d\sigma_t}{\sigma_t} &= \left( \beta (a - \log \sigma_t) + \frac{1}{2} \gamma^2 \right) dt + \gamma d \tilde W_t \\
d \log \sigma_t &= \beta (a - \log \sigma_t) dt + \gamma d \tilde W_t
\end{align*}

\paragraph{Scott PDE}
There are two approaches on how to derive the Scott PDE:
\begin{enumerate}
\item apply Itô's lemma to $C_E = f(S_t,\sigma_t)$ under $\mathbb{Q}$ \\
use the martingale property $\mathbb{E}_\mathbb{Q}[C_E] = r C_E$ \\
(as in the Hull \& White model)
\item apply Itô's lemma to $C_E$ under $\mathbb{P}$ \\
use a continuous time version of the two factor APT (Arbitrage Pricing Theory model) with $\mathbb{E}[dC_E / C_E]$ and $\Phi_t^S = \mu + \delta - r$ \\
equate $\mathbb{E}[\cdot]$
\end{enumerate}
Then:
\begin{align*}
&\frac{1}{2} \sigma^2 x^2 \partial^2_{xx} C_E + \frac{1}{2} \gamma^2 \sigma^2 \partial^2_{\sigma \sigma} C_E + (r - \delta) x \partial_x C_E \\
&\qquad + \sigma (f(\sigma) - \Phi_t^\sigma) \partial_\sigma C_E + \partial_t C_E - r C_E = 0
\end{align*}

\section{Jump Models}

\subsection{Poisson processes}

\paragraph{Standard Poisson process}
A \textit{counting process}:
\begin{itemize}
\item \textit{without explosion} (i.e. $T = \infty$)
\item with \textit{independent increments} \\
i.e. for every $s,t \geq 0$ the RV $N_{t+s} - N_t$ is independent of $\mathcal{F_t^N}$
\item with \textit{stationary increments} \\
i.e. for every $s,t \geq 0$ the RV $N_{t+s} - N_t$ has the same law as $N_s$
\end{itemize}
Important properties:
\begin{align*}
\mathbb{P}[N_t = n] &= e^{-\lambda t} \frac{(\lambda t)^n}{n!} \\
\mathbb{E}[N_t] &= \lambda t \qquad \Var[N_t] = \lambda t \\
\lambda: &\text{ intensity of jumps}
\end{align*}
Characteristic function:
\begin{align*}
\mathbb{E}\left[ e^{i u N_t} \right] &= e^{\lambda t (e^{i u} - 1)}
\end{align*}
Useful properties:
\begin{align*}
\mathbb{E}\left[ e^{\alpha N_t} \right] &= e^{\lambda t (e^\alpha - 1)} \qquad \mathbb{E}\left[ x^{N_t} \right] = e^{\lambda t (x-1)}
\end{align*}

\paragraph{Compensated Poisson process $M_t$}
The following expressions are $\mathbb{F}$-martingales:
\begin{align*}
M_t &:= N_t - \lambda t \\
M_t^2 - \lambda t &= (N_t - \lambda t)^2 - \lambda t
\end{align*}

\paragraph{Itô's formula}
\begin{align*}
dX_t &= h_t dt + f_t dW_t + g_t dM_t \\
dF(t,X_t) &= \partial_t F(t,X_t) dt \\
&\quad + \partial_x F(t,X_{t-}) (dX_t - g_t dN_t) \\
&\quad + \frac{1}{2} \partial^2_{xx} F(t,X_t) d[X]_t \\
&\quad + \left( F(t,X_t) - F(t,X_{t-}) \right) dN_t \\
F(t,X_t) &= F(0,X_0) + \int_0^t \partial_t F(s,X_s) ds \\
&\quad + \int_0^t \partial_x F(s,X_{s-}) dX_s \\
&\quad + \frac{1}{2} \int_0^t \partial^2_{xx} F(s,X_s) f_s^2 ds \\
&\quad + \int_0^t (F(s,X_s) - F(s,X_{s-}) \\
&\qquad - \partial_x F(s,X_{s-} g_s) dN_s
\end{align*}
Assumption: $F$ is a $C^{1,2}$ function on $\mathbb{R}^+ \times \mathbb{R}$, i.e. $\partial_t F, \partial_x F, \partial^2_{xx} F$ exist and are continuous.

Note that:
\begin{itemize}
\item In case of a jump at $t$: $X_t = X_{t-} + g_t$ \\
e.g. if $g_t = X_{t-} \phi$, then $X_t = (1+\phi) X_{t-}$
\item $d[X]_t = f_t^2 dt$
\item If $h_t = f_t = 0$ and $g_t = \phi$, \\
then $S_t = S_0 (1 + \phi)^{N_t}$.
\end{itemize}
Example:
\begin{align*}
\frac{dS_t}{S_{t-}} &= b dt + \sigma dW_t + \phi dM_t \\
d \log S_t &= \left( b - \frac{1}{2} \sigma^2 - \lambda \phi \right) dt \\
& \qquad + \sigma dW_t + \log(1+\phi) dN_t \\
S_t &= S_0 \underbrace{e^{b t}}_\text{drift} \underbrace{e^{\sigma W_t - \frac{1}{2} \sigma^2}}_\text{martingale} \underbrace{e^{\log(1+\phi) N_t - \lambda \phi t}}_\text{martingale}
\end{align*}

\paragraph{Girsanov's theorem for Poisson processes}
Let $L_t$ be the positive exponential martingale solution of:
\begin{align*}
dL_t &= L_{t-} \phi dM_t, & L_t = e^{\log(1+\phi)N_t - \lambda \phi t}
\end{align*}
Let $\mathbb{Q}$ be the probability measure defined by:
\begin{align*}
\left. \frac{d\mathbb{Q}}{d\mathbb{P}} \right|_{\mathcal{F}_t} &= L_t
\end{align*}
Then it holds:
\begin{itemize}
\item under $\mathbb{P}$:
\begin{itemize}
\item $M_t$ is a $\mathbb{P}$-martingale
\item $(N_t)_{t \geq 0}$ is a $\mathbb{P}$-Poisson process of intensity $\lambda$
\end{itemize}
\item under $\mathbb{Q}$:
\begin{itemize}
\item $M_t^\phi = M_t - \phi \lambda t = N_t - (1+\phi) \lambda t$ is a $\mathbb{Q}$-martingale
\item $(N_t)_{t \geq 0}$ is a $\mathbb{Q}$-Poisson process of intensity $(1+\phi)\lambda$
\end{itemize}
\item for the RV $\xi$ that:
\begin{align*}
\mathbb{E}_\mathbb{Q}[\xi] = \mathbb{E}_\mathbb{P}\left[ \xi \cdot \left. \frac{d\mathbb{Q}}{d\mathbb{P}} \right|_{\mathcal{F}_t} \right]
\end{align*}
(i.e. as in the "normal" Girsanov theorem)
\end{itemize}

\subsection{Lévy processes}

\paragraph{Lévy process}
An $\mathbb{R}^d$-valued process $X$:
\begin{itemize}
\item s.t. $X_0 = 0$
\item with \textit{independent increments} \\
i.e. for every $s,t \geq 0$ the RV $X_{t+s} - X_t$ is independent of $\mathcal{F}_t^X$
\item with \textit{stationary increments} \\
i.e. for every $s,t \geq 0$ the RVs $X_{t+s} - X_t$ and $X_s$ have the same law
\item that is \textit{continuous in probability} \\
i.e. for fixed $t$, $\mathbb{P} \left[ | X_t - X_u | > \epsilon \right] \to 0$ when $u \to t$ for every $\epsilon > 0$.
\end{itemize}

\paragraph{Lévy exponent}
If $\mathbb{E} \left[ \exp(k X_1) \right] < \infty$ for any $k$, then the Lévy exponent $\psi$ on $[0,\infty)$ of the Lévy process $X$ is defined as:
\begin{align*}
\mathbb{E} \left[ \exp(k X_1) \right] &= \exp (\psi(k))
\end{align*}
Note that the process $\left( e^{k X_t - t \psi(k)} \right)_{t \geq 0}$ is a martingale for any $k$ s.t. $\psi(k) = \log \mathbb{E} \left[ e^{k X_1} \right] < \infty$

\paragraph{Put-Call symmetries}
\begin{itemize}
\item dynamics under the domestic risk-neutral probability $\mathbb{Q}$
\begin{align*}
\frac{dS_t}{S_{t-}} &= (r-\delta) dt + \sigma dW_t + \phi dM_t \\
S_t &= S_0 e^{(r - \delta) t} e^{\sigma W_t - \frac{1}{2} \sigma^2} e^{\log(1+\phi) N_t - \lambda \phi t}
\end{align*}
\item European put-call symmetry
\begin{align*}
&P_E \left( x,K,r,\delta,\sigma,\phi,\lambda \right) \\
&\qquad = C_E \left( K,x,\delta,r,\sigma,\frac{-\phi}{1+\phi},\lambda(1+\phi) \right)
\end{align*}
\item American put-call symmetry
\begin{align*}
&P_A \left( x,K,r,\delta,\sigma,\phi,\lambda \right) \\
&\quad = K x \cdot C_A \left( \frac{1}{x},\frac{1}{K},\delta,r,\sigma,\frac{-\phi}{1+\phi},\lambda(1+\phi) \right)
\end{align*}
\item Symmetry for exercise boundaries
\begin{align*}
b_p \left( r,\delta,\phi,\lambda \right) \cdot b_c \left( \delta,r,\frac{-\phi}{1+\phi},\lambda(1+\phi) \right) = K^2
\end{align*}
\end{itemize}

\paragraph{Hitting times}
\begin{itemize}
\item dynamics of the underlying
\begin{align*}
S_t &= S_0 e^{ \left( b - \phi \lambda - \frac{1}{2} \sigma^2 \right) t + \sigma W_t + \log(1+\phi) N_t} \\
&= S_0 e^{X_t}
\end{align*}
\item passage times
\begin{align*}
T_L(S) &= \inf \lbrace t \geq 0 : S_t \geq L \rbrace \\
T_\mathcal{L}(X) &= T_L(S) = \inf \lbrace t \geq 0 : X_t \geq \mathcal{L} \rbrace \\
&\qquad \text{with } \mathcal{L} = \log \frac{L}{S_0}
\end{align*}
\item Laplace transform
\begin{align*}
\mathbb{E} \left[ e^{-u T_\mathcal{L}} \right] &=
\begin{cases}
e^{-\psi^{-1}(u) \mathcal{L}} & : \mathcal{L} > 0 \\
1 & : \text{otherwise}
\end{cases}
\end{align*}
with $e^{-\psi^{-1}(u) \mathcal{L}}$ the positive root of $\psi(k) = u$.
\item Overshoot: if the jump size is strictly positive (i.e. $\phi > 0$), there is a non-zero probability that $X_{T_\mathcal{L}}$ is strictly greater than $\mathcal{L}$, i.e.
\begin{align*}
\mathbb{P}[X_{T_\mathcal{L}} > \mathcal{L}] > 0
\end{align*}
Then the overshoot is defined as
\begin{align*}
O_\mathcal{L} &= X_{T_\mathcal{L}} - \mathcal{L}
\end{align*}
If the the jump size is strictly negative (i.e. $-1 < \phi < 0$), then $O_\mathcal{L} = 0$, i.e. $X_t$ is continuous at the boundary.
\end{itemize}

\paragraph{Set of risk-neutral probability measures}
\begin{itemize}
\item dynamics of the underlying
\begin{align*}
\frac{dS_t}{S_{t-}} &= b dt + \sigma dW_t + \phi dM_t \\
S_t &= S_0 e^{ \left( b - \phi \lambda - \frac{1}{2} \sigma^2 \right) t + \sigma W_t + \log(1+\phi) N_t} \\
R(t) &= e^{-r t}
\end{align*}
\item martingale condition: \\
$d(R S)_t$ is a $\mathbb{P}^{\psi, \gamma}$-martingale, i.e.
\begin{align*}
\frac{d(R S)_t}{R_t S_t^-} &= \sigma d \hat W_t  + \phi d \hat M_t
\end{align*}
\item \emph{set $\mathcal{Q}$ of EMMs} defined by:
\begin{align*}
\left. \frac{\mathbb{P}^{\psi, \gamma}}{\mathbb{P}} \right|_{\mathcal{F}_t} &= L_t^{\psi, \gamma} = L_t^\psi(W) L_t^\gamma(M)
\end{align*}
with
\begin{align*}
L_t^\psi(W) &= e^{\psi W_t - \frac{1}{2} \psi^2 t} = \mathcal{E}(\psi W)_t \\
L_t^\gamma(M) &= e^{\log(1+\gamma) N_t - \lambda \gamma t}
\end{align*}
and the constraint
\begin{align*}
b - r + \sigma \psi + \lambda \psi \gamma = 0
\end{align*}
\item $\mathbb{P}^{\psi, \gamma}$-martingales:
\begin{align*}
\hat W_t &= W_t - \psi t \\
\hat M_t &= M_t -  \lambda \gamma t = N_t - \lambda(1+\gamma) t
\end{align*}
\end{itemize}

\section{Real Options}

\subsection{Optimal entry}

\paragraph{Setting}
\begin{itemize}
\item a firm can invest at any time $t$ the investment sum $K_t$ to install a project which generates the sum of expected discounted future net cash-flows $V_t$
\item the investment is irreversible
\item both $K_t$ and $V_t$ are stochastic
\item the maturity is infinite
\end{itemize}

\paragraph{Dynamics}
\begin{itemize}
\item Historical probability $\mathbb{P}$:
\begin{align*}
\frac{dV_t}{V_t} &= \alpha_1 dt + \sigma_1 dW_t \\
\frac{dK_t}{K_t} &= \alpha_2 dt + \sigma_2 dB_t
\end{align*}
where the two $\mathbb{P}$-BM $W_t,B_t$ are correlated with $\rho$.
\item Risk-neutral probability $\mathbb{Q}$:
\begin{align*}
\frac{d(V_t/K_t)}{V_t/K_t} &= (\alpha_1 - \alpha_2) dt + \Sigma dZ_t \\
&\text{with } \Sigma = \sqrt{\sigma_1^2 + \sigma_2^2 - 2 \rho \sigma_1 \sigma_2}
\end{align*}
where $Z_t$ a $\mathbb{Q}$-BM.
\end{itemize}

\paragraph{Real option}
\begin{itemize}
\item Supremum
\begin{align*}
C_\text{RO} &= \sup_{\tau \in \mathcal{T}} \mathbb{E}_\mathbb{P} \left[ e^{-r \tau} (V_\tau - K_\tau) \right] \\
&= \sup_{\tau \in \mathcal{T}} \mathbb{E}_\mathbb{P} \left[ e^{-r \tau} K_\tau \left( \frac{V_\tau}{K_\tau} - 1 \right) \right] \\
&= \sup_{\tau \in \mathcal{T}} K_0 \mathbb{E}_\mathbb{Q} \left[ e^{-(r - \alpha_2) \tau} (V_\tau - K_\tau) \right]
\end{align*}
\item Since this problem is in principle a perpetual American call option:
\begin{align*}
C_\text{RO}(V_0,K_0) &= K_0 (L^\ast - 1) \left( \frac{V_0/K_0}{L^\ast} \right)^\epsilon \\
L^\ast &= \frac{\epsilon}{\epsilon - 1}
\end{align*}
with
\begin{align*}
\epsilon &= \sqrt{\left( \frac{\alpha_1 - \alpha_2}{\Sigma^2} - \frac{1}{2} \right)^2 + \frac{2(r-\alpha_2)}{\Sigma^2}} \\
& \qquad - \left( \frac{\alpha_1 - \alpha_2}{\Sigma^2} - \frac{1}{2} \right)
\end{align*}
\end{itemize}

\paragraph{Interpretation}
In the neoclassical framework, it is optimal to invest if expected discounted earnings are higher than expected discounted costs, i.e. if $X_t > 1$. \\
This framework, however, takes also the risk appropriately into account and thus the optimal time to invest is at $T_{L^\ast}$ with $L^\ast > 1$. In other words, this frameworks generally suggests a certain delay compared to the neoclassical framework IOT account for the risk taken.

%\subsection{Optimal entry in presence of competition}

\subsection{Optimal entry and optimal exit}

\paragraph{Setting}
\begin{itemize}
\item no competition
\item decision to \textit{invest} and irreversible decision to \textit{disinvest}
\item decision to \textit{invest} (with entry cost $K_i$) and irreversible decision to \textit{disinvest} (with exit cost $K_d$)
\item corresponds to embedded perpetual American options, i.e. first an American call (investment) and an American put (disinvestment)
\end{itemize}

\paragraph{Parameters}
\begin{description}[style=multiline,leftmargin=1cm,font=\textbf]
\item[$K_i$] investment cost
\item[$K_d$] disinvestment cost
\item[$K$] sum of all future discounted costs
\item[$\alpha < r$] drift strictly smaller than risk-free rate
\end{description}

\paragraph{Supremum}
\begin{align*}
& VF(S_0) = C_A(S_0) + P_A(S_0) \\
&\quad = \sup_{L_i, L_d} \phi(L_i) \mathbb{E} \left[ e^{-r T_{L_i}} \right] + \psi(L_d) \mathbb{E} \left[ e^{-r T_{L_d}} \right]
\end{align*}
with
\begin{align*}
\phi(L_i) &= \frac{L_i}{r-\alpha} - K - K_i \\
\psi(L_d) &= K - \frac{L_d}{r-\alpha} - K_d
\end{align*}

\paragraph{Laplace transforms}
\begin{align*}
\mathbb{E} \left[ e^{-r T_i} \right] &= \left( \frac{S_0}{L_i} \right)^{\gamma_1} \\
\mathbb{E} \left[ e^{-r (T_d - T_i)} \right] &= \left( \frac{L_i}{L_d} \right)^{\gamma_2}
\end{align*}
with:
\begin{align*}
\gamma_{1,2} &= \frac{-\theta \pm \sqrt{2r + \theta^2}}{\sigma}, \quad \theta = \frac{\alpha - \frac{1}{2}\sigma^2}{\sigma}
\end{align*}

\paragraph{Interpretation}
The possibility to disinvest gives the firm incentives to invest earlier than in the irreversible investment case.

\columnbreak

\section{Systemic Risk}

\paragraph{Introduction}
\begin{itemize}
\item notional value of all derivatives (globally) corresponds to app. 12 times global GDP
\item if derivatives were only used for hedging, notional value would amount to app. 30-40 \% of global GDP
\item conflict of interest for rating agencies: companies pay for their ratings \ldots
\end{itemize}

\paragraph{CDS (Credit Default Swaps)}
\begin{itemize}
\item only half of the CDS in the USA actually covered risks, the other half served only speculative purposes
\item thus, banks can actually benefit more from financial distress of their client companies
\item \emph{ISDA} (International Swaps \& Derivatives Association): makes finally the decision whether a credit event takes place or not (which is relevant for CDS), but: members of the voting board of ISDA are the major banks who also purchased the CDS
\end{itemize}

\paragraph{Food speculation}
\begin{itemize}
\item \emph{Impact of food speculation on prices:}
\begin{itemize}
\item Paradox: usually, the futures prices converge to the spot price at maturity
\item however: on the food market it is vice-versa, i.e. the spot prices converge to the futures prices
\end{itemize}
\item demand for commodities was (is) dominated by speculators (between 65--80 \%)
\end{itemize}

\paragraph{High-Frequency Trading (HFT)}
\begin{itemize}
\item the economy does not work in milliseconds, but takes days, weeks, years to adapt \ldots
\item \emph{Front trading:}
\begin{itemize}
\item buying a stock just before an investor wants to buy it IOT sell the stock to the same investor for a slightly higher price
\item this is a kind of a tax on investors who do not have access to HFT
\item of course this is illegal (kind of insider trading)
\end{itemize}
\item counter-regulation: e.g. micro-tax on electronic payments
\end{itemize}

\paragraph{Fiscal arbitrage}
\begin{itemize}
\item e.g. Goldman Sachs created a structured product (basket) IOT allow Greece to convert debt in USD into EUR at an arbitrary higher exchange rate \\
result: high commission for Goldman Sach and a mean for the Greek government to hide debt
\end{itemize}

\paragraph{Insider trading}
\begin{itemize}
\item criteria:
\begin{enumerate}
\item surprisingly high volume (and open interest)
\item impressive profits within a short period (e.g. 200--500 \%)
\item transactions without hedging
\end{enumerate}
\item insider trading for stocks: can be justified since private information is revealed
\item insider trading for options: cannot be justified since no information is actually revealed (e.g. open interest is not considered in the BS formula)
\end{itemize}

\section{Environmental Finance}

\paragraph{Environmental Finance}
\begin{itemize}
\item part of environmental economics, exploits financial instruments IOT deal with ecological issues \\
e.g. land use planning, natural resource preservation, urban growth issues
\item focuses on financial and quantitative issues, proposed by environmental economists
\item quantitative analysis of the impact of market-based environmental policies \\
e.g. European Emission Trading Scheme
\end{itemize}

\paragraph{Global warming}
\begin{itemize}
\item \emph{perspectives on how to tackle global warming:}
\begin{itemize}
\item economics: introduce taxes
\item finance: introduce a market
\end{itemize}
\item \emph{ecological footprint:} measures how much land and water area a human population requires to produce the resources it consumes and to absorb its wastes, using prevailing technology \\
e.g. in 2012: humanity used 1.8 planets (i.e. 2 yrs of regeneration)
\item \emph{CO\textsubscript{2} concentration in the atmosphere:}
\begin{itemize}
\item since industrialisation, CO\textsubscript{2} concentration has been continuously increasing
\item today's concentration is 400ppm, while the safe upper limit might rather be 350 ppm
\end{itemize}
\item \emph{CO\textsubscript{2} emissions}
\begin{itemize}
\item if all CO\textsubscript{2} emissions were globally stopped, it would take 1,000 yrs to decrease global temperature by 1 K
\item biggest carbon emissioning states: USA and China
\item average CO\textsubscript{2} emissions per capita, e.g. CHE 4.6 t (in 2011)
\item goal: 1 t p.c. and p.a.
\end{itemize}
\item \emph{temperature increase:}
\begin{itemize}
\item current projection if no actions are taken: +4.5 K (catastrophic outcome \ldots)
\item reconstruction of temeprature curves: data from ice layers, different models provide different reconstructions of temperature curves
\end{itemize}
\item \emph{consequences of global warming:}
\item more extreme events (e.g. insurance industry)
\item adaption costs (e.g. in developping countries IOT mitigate effects)
\end{itemize}

\paragraph{Kyoto protocol}
\begin{itemize}
\item Greenhouse gas emissions (GHG) of developed countries to be reduced by 5.2 \% from 1990 to 2012 (Europe: 8 \%) \\
However, in 2011, global emissions were 42 \% higher than in 1990
\item \emph{market-based mechanisms:} emission permits
\item \emph{project-based mechanisms:} certificates of emission reduction
\end{itemize}

\paragraph{EU ETS}
\begin{itemize}
\item started in 2005 and covers 5 main sectors: pulp \& paper, steel, cement, energy production, ore \& mining
\item each relevant company received a pre-specified amount of permits (emission allowances)
\item \emph{price development:}
\begin{itemize}
\item start with a high price, decline after 2005, crashed by 2010 (partly also due to the financial crisis)
\item Brussels had no clue about CO\textsubscript{2}-emissions, thus companies were polled, but they initially reported way too much
\item Stiglitz: CO\textsubscript{2} price per t will increase to 100 EUR at some point \\
(40 EUR per t is required as an incentive for sustainable development)
\end{itemize}
\item \emph{issues:}
\begin{itemize}
\item allocation criteria: grandfathering vs. auctioning
\item windfall profits, e.g. energy sector can pass opportunity costs of emission permits on the end-consumer
\item duration of the scheme: the longer, the better?
\item relevant sectors to include, e.g. what about aviation, transportation, households?
\end{itemize}
\item \emph{emission allowance:}
\begin{itemize}
\item a limited, transferable right to emit one ton of an offending gas or carbon dioxide equivalent (CO\textsubscript{2}e)
\item goal of the provisions: cost-efficiency
\end{itemize}
\item \emph{marginal cost theory:}
\begin{itemize}
\item price of a permit equal to the marginal cost of abatement, i.e.
\begin{align*}
S_t = MC_t
\end{align*}
i.e. if $S_t > MC_t$, sell permit \& adapt clean technology \\
i.e. if $S_t < MC_t$, buy permit \& adapt clean technology
\item companies should reduce their expected discounted costs:
\begin{align*}
\min_{X_0} \left\lbrace P_0 X_0 + (1 + \eta)^{-1} \mathbb{E} \left[ g(Q) \cdot (P_1 + P_2) \right] \right\rbrace
\end{align*}
\item but: marginal cost theory inconsistent with reality
\item "special" stochastic process $S_t$: on $[0,T)$ a BM, but at $T$ either $0$ or $P$ \ldots
\item unknowns: $X_1$ (emissions of company 1), $X_2$ (emissions of company 2), $S_T$ (price of CO\textsubscript{2}-certificates at $T$)
\end{itemize}

\end{itemize}

\columnbreak
\begin{center}
\textit{intentionally left blank}
\end{center}
\vfill
\section*{Notation}

\begin{description}[style=multiline,leftmargin=1cm,font=\textbf]
\item[$\varphi$] PDF of the standard normal distribution
\item[$\Phi$] CDF of the standard normal distribution
\end{description}

\section*{Abbreviations}

\begin{description}[style=multiline,leftmargin=1cm,font=\textbf]
%\item[a.a.] almost all
%\item[a.s.] almost surely
\item[BM] Brownian motion
\item[CDF] cumulative distribution function
\item[e.g.] exempli gratia
\item[EMM] equivalent martingale measure
\item[i.e.] id est
%\item[iff] if and only if
\item[IOT] in order to
\item[p.a.] per annum
\item[p.c.] per capita
\item[PDE] partial differential equation
\item[PDF] probability density function
%\item[RCLL] right-continuous with left limits
\item[RV] random variable
%\item[SDE] stochastic differential equation
\item[s.t.] such that
%\item[w.r.t.] with respect to
\end{description}

\section*{Disclaimer}

\begin{itemize}
\item This summary is work in progress, i.e. neither completeness nor correctness of the content are guaranteed by the author.
\item This summary may be extended or modified at the discretion of the readers.
\item Source: Lecture \SummaryTitle, \SummarySemester, UZH (lecture notes, script, exercises and literature). Copyright of the content is with the lecturers.
\item The layout of this summary is based on the summaries of several courses of the BSc ETH ME from Jonas LIECHTI.
\end{itemize}

\end{multicols*}

\end{document}