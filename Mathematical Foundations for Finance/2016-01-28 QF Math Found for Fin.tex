\documentclass[a4paper,landscape,8pt,fleqn]{scrartcl}
\usepackage[ngerman]{babel}
\usepackage[utf8]{inputenc}
\usepackage[a4paper, landscape, margin=1.1cm]{geometry}
\usepackage{latexsym}
\usepackage{multicol}
\usepackage{amsmath}
\usepackage{amsfonts}
\usepackage{amssymb}
\usepackage{array}
\usepackage{graphicx}
\usepackage{booktabs}
\usepackage{empheq}			% emphasize (box) equations
\usepackage{float}				% add H as an option for floats
\usepackage{parskip}			% add no indentation to new paragraphs
\usepackage{enumitem}		% description lists
\usepackage{fancyhdr}
\usepackage{lastpage}
\usepackage{framed}

\pagestyle{plain}
\columnsep 30pt
\columnseprule .4pt

\setlength{\mathindent}{0.1cm}

\newcommand*\widefbox[1]{\fbox{\hspace{2em}#1\hspace{2em}}}		% required for boxing several lines of equations at once

\renewcommand*{\familydefault}{\sfdefault}		% set font to default sans-serif

\renewcommand{\labelitemi}{\tiny$\blacksquare$}		% change symbol of itemized lists
\setlist[itemize]{leftmargin=0.4cm}								% reduce indentation of itemized lists
\renewcommand{\labelenumi}{(\roman{enumi})}			% change counter of enumerated lists
\setlist[enumerate]{leftmargin=0.4cm}							% reduce indentation of enumerated lists

\renewcommand{\arraystretch}{1}
\renewcommand{\emph}[1]{\textbf{#1}}

%\allowdisplaybreaks	% equations can be split on two pages/columns

\graphicspath{ {images/} }

\pagestyle{fancy}
\fancyhead{}
\setlength{\headheight}{0pt}
\setlength{\footheight}{14pt}
\renewcommand{\headrulewidth}{0pt}
\renewcommand{\footrulewidth}{0.5pt}
\lfoot{Fabian MARBACH}
\cfoot{p \thepage\ / \pageref{LastPage}}
\rfoot{Mathematical Foundations for Finance}

\usepackage{parskip}	% add no indentation to new paragraphs

\makeatletter
\renewcommand{\section}{\@startsection{section}{1}{0mm}%
{-2\baselineskip}{0.8\baselineskip}%
{\hrule depth 0.2pt width\columnwidth\hrule depth1.5pt
width0.25\columnwidth\vspace*{1.2em}\Large\bfseries}}
\makeatother

\makeatletter
\renewcommand{\subsection}{\@startsection{subsection}{1}{0mm}%
{-2\baselineskip}{0.8\baselineskip}%
{\hrule depth 0.2pt width\columnwidth\hrule depth0.75pt
width0.25\columnwidth\vspace*{1.2em}\large\bfseries}}
\makeatother

\makeatletter
\renewcommand{\subsubsection}{\@startsection{subsubsection}{1}{0mm}%
{-2\baselineskip}{0.8\baselineskip}%
{\hrule depth 0.2pt width\columnwidth\vspace*{1.2em}\normalsize\bfseries}}
\makeatother

\newcommand{\Mx}[1]{\begin{bmatrix}#1\end{bmatrix}}
\newcommand{\dd}[2]{\frac{\text{d}#1}{\text{d}#2}}
\newcommand{\DD}[2]{\frac{\text{D}#1}{\text{D}#2}}
\newcommand{\deidei}[2]{\frac{\partial#1}{\partial#2}}
\newcommand{\Lbrace}[1]{\left\{\begin{array}{ll}#1\end{array}\right.} % left brace with text

% Declare mathematical operators
\DeclareMathOperator{\erf}{erf}					% error function
\DeclareMathOperator{\Var}{Var}				% variance
\DeclareMathOperator{\Varn}{Varn}			% variation
\DeclareMathOperator{\QV}{QV}				% quadratic variation
\DeclareMathOperator{\Cov}{Cov}				% covariance
\DeclareMathOperator{\Ber}{Ber}				% Bernoulli distribution
\DeclareMathOperator{\Bin}{Bin}				% Binomial distribution
\DeclareMathOperator{\Geom}{Geom}		% Geometric distribution
\DeclareMathOperator{\Poi}{Poi}					% Poisson distribution
\DeclareMathOperator{\Gammaa}{Gamma}	% Gamma distribution
\DeclareMathOperator{\Cau}{Cau}				% Cauchy distribution
\DeclareMathOperator{\Adj}{Adj}				% Adjoint
\DeclareMathOperator{\Bias}{Bias}				% Bias of an estimator
\DeclareMathOperator{\rank}{rank}				% Bias of an estimator


\begin{document}
\part*{Summary: Mathematical Foundations for Finance}
Fabian MARBACH, Autumn Semester 2015/16
\begin{multicols*}{3}
%\tableofcontents
%\end{multicols}
%{\vspace*{0.3cm}}
%{\hrule depth 0.2pt}
%{\vspace*{0.3cm}}
%\begin{multicols}{2}
\raggedcolumns
\newpage

\section{Financial Markets in Discrete Time}

\subsection{Basic setting}

\paragraph{Basic setting of the market}

\begin{itemize}
\item \emph{Probability space:} $(\Omega, \mathcal{F}, \mathbb{P})$
\item Finite discrete time horizon: $k = 0,1, \ldots, T$
\item Flow of information over time: \emph{filtration} $\mathbb{F} = (\mathcal{F}_k)_{k=0,1, \ldots,T}.$ \\
This is a family of $\sigma$-fields $\mathcal{F}_k \subseteq \mathcal{F}$ which is increasing.
\item An ($\mathbb{R}^d$-valued) \emph{stochastic process} in discrete time: $X = (X_k)_{k=0,1, \ldots, T}$ of ($\mathbb{R}^d$-valued) RVs which are all defined on the same probability space. \\
This describes the random evolution over time of $d$ quantities.
\item A stochastic process is called \emph{adapted} (w.r.t. $\mathbb{F}$) if each $X_k$ is $\mathcal{F}_k$-measurable, i.e. observable at time $k$.
\item A stochastic process is called \emph{predictable} (w.r.t. $\mathbb{F}$) if each $X_k$ is even $\mathcal{F}_{k-1}$-measurable.
\end{itemize}

\paragraph{Frictionless financial market}

\begin{itemize}
\item no transaction cost
\item short-selling allowed
\item investors are small, i.e. their trading does not affect stock prices
\end{itemize}

\subsection{Basic processes}

\paragraph{Trading strategy $\varphi$}

\begin{itemize}
\item A trading stragey is an $\mathbb{R}^d$-valued stochastic process $\varphi = (\varphi^0, \vartheta)$.
\item $\varphi^0 = (\varphi_k^0)_{k=0,1, \ldots, T}$ denotes the riskless asset. \\
$\varphi^0$ is $\mathbb{R}$-valued and adapted.
\item $\vartheta = (\vartheta_k)_{k=0,1, \ldots, T}$ denotes the $d$ risky assets. \\
$\vartheta$ is $\mathbb{R}^d$-valued and predictable.
\item Initial value: $\varphi = (\varphi^0,\vartheta_0 \equiv 0)$ \\
(since there is no trading before time 0, i.e. investors start out without any shares)
\item A trading strategy describes a dynamically evolving portfolio in the $d+1$ basic assets available for trade.
\end{itemize}

\paragraph{Value process $V$}

\begin{itemize}
\item $V(\varphi) = (V_k(\varphi))_{k=0,1, \ldots, T}$ denotes the discounted value process of the strategy $\varphi$ and is given by
\begin{align*}
V_k(\varphi) &:= \underbrace{\varphi_k^0 S_k^0}_{\text{bank account}} + \underbrace{\vartheta_k \cdot S_k}_{\text{portfolio}} = \varphi_k^0 + \sum_{i=1}^d \vartheta_k^i S_k^i
\end{align*}
\item $V$ is $\mathbb{R}$-valued and adapted.
\item Initial value: $V_0(\varphi) = \varphi_0^0 = C_0(\varphi)$.
\end{itemize}

\paragraph{Cost process $C$}

\begin{itemize}
\item $(C_k(\vartheta))_{k=0,1,\ldots, T}$ denotes the discounted cost process associated to $\varphi$:
\begin{align*}
C_k(\varphi)& := V_k(\varphi) - G_k(\varphi)
\end{align*}
\item By construction, $C_k(\varphi)$ describes the cumulative (total) costs for the strategy $\varphi$ on the time interval $[0,k]$.
\item Incremental cost:
\begin{align*}
\Delta C_{k+1}(\varphi) :&= \underbrace{\varphi_{k+1}^0 - \varphi_k^0}_\text{bank account} + \underbrace{\sum_{i=1}^d \left( \vartheta^i_{k+1} - \vartheta^i_k \right) S^i_k}_\text{portfolio}
\end{align*}
\end{itemize}

\paragraph{Gains process $G$}

\begin{itemize}
\item $(G_k(\vartheta))_{k=0,1,\ldots, T}$ denotes the discounted gains process associated to $\vartheta$:
\begin{align*}
G_k(\vartheta) :&= \sum_{j=1}^k \vartheta_j \cdot \Delta S_j
\end{align*}
\item $G$ is $\mathbb{R}$-valued and adapted.
\end{itemize}

\paragraph{Self-financing strategy}

\begin{itemize}
\item A trading strategy $\varphi = (\varphi^0, \vartheta)$ is called self-financing if its cost process $C(\varphi)$ is constant over time.
\item A self-financing strategy $\varphi = (\varphi^0, \vartheta)$ is uniquely determined by its initial wealth $V_0$ and its risky asset component $\vartheta$. \\
In particular, any pair $(V_0, \vartheta)$ specifies in a unique way a self-financing strategy. \\
If $\varphi = (\varphi^0, \vartheta)$ is self-financing, then $(\varphi_k^0)_{k=1,\ldots, T}$ is automatically predictable.
\item It then holds for the corresponding (incremental) \emph{cost process:}
\begin{align*}
\Delta C_{k+1}(\varphi) &= \varphi_{k+1}^0 - \varphi_k^0 + (\vartheta_{k+1} - \vartheta_k) \cdot S_k = 0 \\
C(\varphi) &= C_0(\varphi) = V_0(\varphi) = \varphi_0^0
\end{align*}
\item It then holds for the corresponding \emph{value process:}
\begin{align*}
V_k(\varphi) &= V_0(\varphi) + G_k(\vartheta) = \varphi_0^0 + G_k(\vartheta) \\
&= \varphi_0^0 + \sum_{j=1}^k \vartheta_j \cdot \Delta S_j
\end{align*}
\end{itemize}
Remarks:
\begin{itemize}
\item The notion of a strategy being self-financing is a kind of economic budget constraint.
\item The notion of self-financing is numeraire irrelevant, i.e. it does not depend on the units in which the calculations are done.
\end{itemize}

\paragraph{Admissibility}

\begin{itemize}
\item For $a \in \mathbb{R}, a \geq 0$, a trading strategy $\varphi$ is called \emph{a-admissible} if its value process $V(\varphi)$ is uniformly bounded from below by $-a$, i.e.
\begin{align*}
V_k(\varphi) \geq -a, \quad \mathbb{P}\text{-a.s.}, \quad \forall k \geq 0
\end{align*}
\item A trading strategy is called \emph{admissible} if it is a-admissible for some $a \geq 0$.
\end{itemize}
Remark:
\begin{itemize}
\item An admissible strategy can be interpreted as a strategy having some credit line which imposes a lower bound on the associated value process. So one may make debts, but only within clearly defined limits.
\end{itemize}

\subsection{Properties of the market}

\paragraph{Characterisation of financial markets via EMMs}

The description of a financial market model via EMMs can be summarized as follows:
\begin{itemize}
\item \emph{Existence} of an EMM $\iff$ the market is \emph{arbitrage-free} \\
i.e. $\mathbb{P}_e(S) \neq \emptyset$ by the 1\textsuperscript{st} FTAP
\item \emph{Uniqueness} of the EMM $\iff$ the market is \emph{complete} \\
i.e. $\#(\mathbb{P}_e(S)) = 1$ by the 2\textsuperscript{nd} FTAP
\end{itemize}

\subsubsection{Arbitrage}

\paragraph{1\textsuperscript{st} Fundamental Theorem of Asset Pricing (FTAP)}

\begin{itemize}
\item Consider a financial market model in finite discrete time.
\item Then $S$ is arbitrage-free iff there exists an EMM for $S$, i.e.
\begin{empheq}[box=\widefbox]{align*}
\text{(NA)} \iff \mathbb{P}_e(S) \neq \emptyset
\end{empheq}
\item In other words: If there exists a probability measure $\mathbb{Q} \approx \mathbb{P}$ on $\mathcal{F}_T$ s.t. $S$ is a $\mathbb{Q}$-martingale, then $S$ is arbitrage-free.
\end{itemize}

\paragraph{Arbitrage opportunity}

\begin{itemize}
\item An arbitrage opportunity is an admissible self-financing strategy $\varphi = (0,\vartheta)$ with zero initial wealth, with $V_t(\varphi) \geq 0$, $\mathbb{P}$-a.s. and with $\mathbb{P}[V_T(\varphi) > 0] > 0$.
\item The financial market $(\Omega, \mathcal{F}, \mathbb{F}, \mathbb{P}, S^0, S)$ or shortly $S$ is called \emph{arbitrage-free} if there exist no arbitrage opportunities.
\item The following statements are equivalent:
\begin{enumerate}
\item $S$ is arbitrage-free.
\item There exists no self-financing strategy $\varphi = (0, \vartheta)$ with zero initial wealth and satisfying $V_T(\varphi) \geq 0$, $\mathbb{P}$-a.s. and $\mathbb{P}[V_T(\varphi > 0] > 0$. In other words, $S$ satisfies (NA').
\item For every (not necessarily admissible) self-financing strategy $\varphi$ with $V_0(\varphi) = 0$, $\mathbb{P}$-a.s. and $V_T(\varphi) \geq 0$, $\mathbb{P}$-a.s., we have $V_T(\varphi) = 0$, $\mathbb{P}$-a.s.
\item For the space
\begin{align*}
\mathcal{G}' := \lbrace G_T(\varphi) | \nu \text{ is } \mathbb{R}^d \text{-valued and predictable} \rbrace
\end{align*}
of all final wealths that one can generate from zero initial wealth through self-financing trading, we have
\begin{align*}
\mathcal{G}' \cap \mathcal{L}_+^0(\mathcal{F}_T) = \lbrace 0 \rbrace
\end{align*}
where $\mathcal{L}_+^0(\mathcal{F}_T)$ denotes the space of all nonnegative $\mathcal{F}_T$-measurable RVs.
\end{enumerate}
\item \textit{Interpretation:} Absence of arbitrage is a natural economic/financial requirement for a reasonable model of a financial market, since there cannot exist ''money pumps'' (at least not for long).
\end{itemize}

\subsubsection{Completeness}

\paragraph{2\textsuperscript{nd} Fundamental Theorem of Asset Pricing (FTAP)}

\begin{itemize}
\item Consider a financial market model in finite discrete time and assume that $S$ is arbitrage-free, $\mathcal{F}_0$ is trivial and $\mathcal{F}_T = \mathcal{F}$.
\item Then $S$ is complete iff there is a unique EMM for $S$, i.e.
\begin{empheq}[box=\widefbox]{align*}
\text{(NA)} + \text{completeness} \iff \#(\mathbb{P}_e(S)) = 1
\end{empheq}
\end{itemize}

\paragraph{Completeness of the market}

\begin{itemize}
\item A financial market model (in finite discrete time) is called complete if \textit{every payoff $H \in \mathcal{L}_+^0(\mathcal{F}_T)$ is attainable.}
\item Otherwise it is called \textit{incomplete}.
\end{itemize}
Remark:
\begin{itemize}
\item If a financial market in discrete time is complete, then $\mathcal{F}_T$ is finite (i.e. completeness is quite restrictive).
\end{itemize}

\subsection{Pricing of contingent claims $H$}

\paragraph{Attainability}

\begin{itemize}
\item A payoff $H \in \mathcal{L}_+^0(\mathcal{F}_T)$ is called attainable if there exists an admissible self-financing strategy $\varphi = (V_0, \vartheta)$ with $V_T(\varphi) = H$ $\mathbb{P}$-a.s.
\item The stratety $\varphi$ is then said to \emph{replicate} $H$ and is called a \emph{replicating strategy} for $H$.
\end{itemize}

\paragraph{Valuation in complete markets (martingale pricing approach)}

\begin{itemize}
\item Consider a financial market in finite discrete time and suppose that $S$ is arbitrage-free and complete and $\mathcal{F}_0$ is trivial.
\item Then for every payoff $H \in \mathcal{L}_+^0(\mathcal{F}_T)$, there is a unique price process $V^H = (V_k^H)_{k=0,1,\ldots,T}$ which admits no arbitrage.
\item $V^H$ is given by:
\begin{empheq}[box=\widefbox]{align*}
V_k^H &= \mathbb{E}_\mathbb{Q} \left[ H | \mathcal{F}_k \right] = V_k(V_0,\vartheta)
\end{empheq}
for $k=0,1,\ldots,T$, for any EMM $\mathbb{Q}$ for $S$ and for any replicating strategy $\varphi = (V_0,\vartheta)$ for $H$.
\end{itemize}

\paragraph{Characterization of attainable payoffs}

\begin{itemize}
\item Consider a financial market in finite discrete time and suppose that $S$ is arbitrage-free and $\mathcal{F}_0$ is trivial.
\item For any payoff $H \in \mathcal{L}_+^0(\mathcal{F}_T)$, the following are equivalent:
\begin{enumerate}
\item $H$ is \textit{attainable}.
\item $\sup\limits_{\mathbb{Q} \in \mathbb{P}_e(S)} \mathbb{E}_\mathbb{Q}[H] < \infty$ is attained in some $\mathbb{Q}^\ast \in \mathbb{P}_e(S)$, i.e. the supremum is finite and a maximum. \\
In other words, we have $\sup\limits_{\mathbb{Q} \in \mathbb{P}_e(S)} \mathbb{E}_\mathbb{Q}[H] = \mathbb{E}_{\mathbb{Q}^\ast}[H] < \infty$ for some $\mathbb{Q}^\ast \in \mathbb{P}_e(S)$.
\item The mapping $\mathbb{P}_e(S) \rightarrow \mathbb{R}$, $\mathbb{Q} \mapsto \mathbb{E}_\mathbb{Q}[H]$ is constant, i.e. $H$ has the same and finite expectation under all EMMs $\mathbb{Q}$ for $S$.
\end{enumerate}
\item Remark: Note that not all of these relationships necessarily hold for financial markets in infinite discrete time or continuous time.
\end{itemize}

\paragraph{Approach to valuing and hedging payoffs}

For a given payoff $H$ in a financial market in finite discrete time (with $\mathcal{F}_0$ trivial):

\begin{enumerate}
\item Check if $S$ is arbitrage-free by finding at least one EMM $\mathbb{Q}$ for $S$.
\item Find all EMMs $\mathbb{Q}$ for $S$.
\item Compute $\mathbb{E}_\mathbb{Q}[H]$ for all EMMs $\mathbb{Q}$ for $S$ and determine the supremum of $\mathbb{E}_\mathbb{Q}[H]$ over $\mathbb{Q}$.
\item If the supremum is finite and a maximum, i.e. attained in some $\mathbb{Q}^\ast \in \mathbb{P}_e(S)$, then $H$ is attainble and its price process can be computed as $V_k^H = \mathbb{E}_\mathbb{Q}[H | \mathcal{F}_k]$, for any $\mathbb{Q} \in \mathbb{P}_(S)$. \\
If the supremum is not attained (or, equivalently for finite discrete time, there is a pair of EMMs $\mathbb{Q}_1, \mathbb{Q}_2$ with $\mathbb{E}_{\mathbb{Q}_1}[H] \neq \mathbb{E}_{\mathbb{Q}_2}[H]$), then $H$ is not attainable.
\end{enumerate}

\paragraph{Invariance of the risk-neutral pricing method under a change of numéraire}

\begin{itemize}
\item The risk-neutral pricing method is invariant under a change of numéraire, i.e. all assets can be priced under a risk-neutral method independent of the chosen asset used for discounting.
\item Denote with $Q^{\ast \ast}$ the EMM for $\hat S^0 := \frac{\tilde S^0}{\tilde S^1}$. \\
Denote with $Q^\ast$ the EMM for $S^1 = \frac{\tilde S^1}{\tilde S^0}$.
\item Then it holds for a financial market $(\tilde S^0, \tilde S^1)$ and an undiscounted payoff $\tilde H \in L^0_+(\mathcal{F}_T)$ that:
\begin{align*}
\tilde S^0_k \mathbb{E}_{\mathbb{Q}^\ast} \left[ \left. \frac{\tilde H}{\tilde S^0_T} \right\vert \mathcal{F}_k \right]
&= \tilde S^1_k \mathbb{E}_{\mathbb{Q}^{\ast \ast}} \left[ \left. \frac{\tilde H}{\tilde S^1_T} \right\vert \mathcal{F}_k \right]
\end{align*}
\end{itemize}

\paragraph{EMMs in submarkets}

\begin{itemize}
\item If a market $(S^0, S^1, \ldots, S^k)$ is (NA), i.e. there exists an EMM $\mathbb{Q}$, then this EMM $\mathbb{Q}$ is also an EMM for all submarkets. \\
(e.g. for $(S^k, S^i), k \neq i$, for $(S^k, S^i, S^j), k \neq i \neq j$ etc.)
\item If there exists a EMM $\mathbb{Q}^j$ for a submarket $(S^0, S^j)$ which is not also an EMM for another submarket $(S^0, S^k)$, $j \neq k$, then the whole market $(S^0, S^1, \ldots, S^k)$ is not (NA), i.e. it admits arbitrage.
\end{itemize}

\subsection{Multiplicative model}

\begin{itemize}
\item Suppose that we start with the RVs $r_1, \ldots, r_T$ and $Y_1, \ldots, Y_T$.
\item Define the \emph{bank account/riskless asset} by:
\begin{align*}
\tilde S_k^0 := \prod_{j=1}^k (1+r_j), \qquad \frac{\tilde S_k^0}{\tilde S_{k-1}^0} = 1 + r_k, \qquad \tilde S_0^0 = 1
\end{align*}
Remarks:
\begin{itemize}
\item $\tilde S_k^0$ is $\mathcal{F}_{k-1}$-measurable (i.e. predictable).
\item $r_k$ denotes the rate for $(k-1,k]$.
\end{itemize}
\item Define the \emph{stock/risky asset} by:
\begin{align*}
\tilde S_k^1 := S_0^1 \prod_{j=1}^k Y_j, \qquad \frac{\tilde S_k^1}{\tilde S_{k-1}^1} = Y_k, \qquad \tilde S_0^1 = \text{const.}, \quad \tilde S_0^1 \in \mathbb{R}
\end{align*}
Remarks:
\begin{itemize}
\item $\tilde S_k^1$ is $\mathcal{F}_k$-measurable (i.e. adapted).
\item $Y_k$ denotes the growth factor for $(k-1,k]$.
\item The rate of return $R_k$ is given by $Y_k = 1 + R_k$.
\end{itemize}
\end{itemize}

\subsubsection{Cox-Ross-Rubinstein (CRR) binomial model}

\paragraph{Assumptions}
\begin{itemize}
\item \textit{Bank account/riskless asset:} \\
Suppose all the $r_k \in \mathbb{R}$ are constant with value $r > -1$. \\
This means that we have the same nonrandom interest rate over each period. \\
Then the bank account evolves as $\tilde S_k^0$ for $k=0,1, \ldots, T$.
\item \textit{Stock/risky asset:} \\
Suppose that $Y_1, \ldots Y_T \in \mathbb{R}$ are independent and only take two values, $1+u$ with probability $p$, and $1+d$ with probability $1-p$ (i.e. all $Y_k$ are i.i.d.). \\
Then the stock prices at each step moves either up (by a factor $1+u$) or down (by a factor $1+d$).
\end{itemize}

\paragraph{Martingale property}

The discounted stock price $\frac{\tilde S^1}{\tilde S^0}$ is a $\mathbb{P}$-martingale iff $r = p u + (1-p) d$.

\paragraph{EMM}

\begin{itemize}
\item In the binomial model, there exists a probability measure $\mathbb{Q} \approx \mathbb{P}$ s.t. $\frac{\tilde S^1}{\tilde S^0}$ is a $\mathbb{Q}$-martingale iff $u > r > d$.
\item In that case, $\mathbb{Q}$ is unique (on $\mathcal{F}_T$) and characterised by the property that $Y_1, \ldots, Y_T$ are i.i.d. under $\mathbb{Q}$ with parameter
\begin{align*}
q^\ast &= \mathbb{Q}[Y_k = 1+u] = \frac{r-d}{u-d} & (\nearrow \text{up}) \\
1-q^\ast &= 1 - \mathbb{Q}[Y_k = 1+d] = \frac{u-r}{u-d} & (\searrow \text{down})
\end{align*}
\end{itemize}

\paragraph{Arbitrage and completeness}

The following statements are equivalent:
\begin{enumerate}
\item $u>r>d$
\item $\exists$ a unique EMM $\mathbb{Q}^\ast$ for $\frac{\tilde S^1}{\tilde S^0}$ (on $\mathcal{F}_T$)
\item The market $S$ is (NA) and complete.
\end{enumerate}

\paragraph{Put-Call parity}

\begin{itemize}
\item Assuming $T=1$, it holds that:
\begin{align*}
V_0^{C(K)} - V_0^{P(K)} = S^1_0 - \frac{K}{1+r}
\end{align*}
\end{itemize}

\paragraph{Pricing binomial contingent claims $H$}

\begin{itemize}
\item Assume time horizon $T=1$, strike $K>0$, and a payoff function $H(x,K): \mathbb{R}_+ \to \mathbb{R}_+$ of a European style contingent claim with strike $K$.
\item $H$ may be a European call function $C(x,K)=(x-K)^+$ \\
or a European put function $P(x,K)=(K-x)^+$.
\item Then $H$ can be replicated using a self-financing strategy $\varphi^{H(K)} = (V_0^{H(K)},\vartheta^{H(K)})$ s.t.
\begin{align*}
V_1(\varphi^{H(K)} &= \frac{H(\tilde H_1^1,K)}{1+r}, \qquad \mathbb{P}-a.s.
\end{align*}
and $\varphi^{H(K)}$ is given by
\begin{align*}
V_0^{H(K)} &= \frac{r-d}{u-d} \frac{H(S_0^1(1+u),K)}{1+r} + \frac{u-r}{u-d} \frac{H(S_0^1(1+d),K)}{1+r} \\
\vartheta_1^{H(K)} &= \frac{H(1+u,K/S_0^1)-H(1+d,K/S_0^1)}{u-d}
\end{align*}
\item Note that this can also be expressed via the martingale pricing approach:
\begin{align*}
V_0^{H(K)} &= \mathbb{E}_\mathbb{Q} \left[ \frac{H(\tilde S_1^1,K)}{1+r} \right]
\end{align*}
where
\begin{align*}
\mathbb{Q}\left[ \tilde S_1^1=S_1^0(1+u) \right] &= q = \frac{r-d}{u-d} \\
\mathbb{Q}\left[ \tilde S_1^1=S_1^0(1+d) \right] &= 1-q = \frac{u-r}{u-d}
\end{align*}
\end{itemize}

\paragraph{Binomial call pricing formula}

\begin{align*}
\tilde V_k^{\tilde H} &= \tilde S_k^1 \mathbb{Q}^{\ast \ast} [W_{k,T} > x] - \tilde K \frac{\tilde S_k^0}{\tilde S_T^0} \mathbb{Q}^\ast [W_{k,T} > x] \\
x &= \frac{\log \frac{\tilde K}{\tilde S_k^1} - (T-k) \log(1+d)}{\log \frac{1+u}{1+d}}
\end{align*}
Remark:
\begin{itemize}
\item This is the discrete analogue of the Black-Scholes formula.
\end{itemize}

\subsubsection{Multinomial model}

\paragraph{EMM}

\begin{itemize}
\item IOT construct an EMM for $S^1$, it needs to hold that:
\begin{align*}
\mathbb{E}_\mathbb{Q}[S_1^1] &= S_0^1 \\
\iff \mathbb{E}_\mathbb{Q}[Y_1] &= 1+r \quad \iff \quad \sum_{k=1}^m q_k (1+y_k) = 1+r
\end{align*}
with the further conditions:
\begin{align*}
\sum_{k=1}^m q_i &= 1, \qquad q_1,\ldots, q_m \in (0,1)
\end{align*}
\end{itemize}

\paragraph{Arbitrage condition}

The following statements are equivalent:
\begin{enumerate}
\item $y_1 < r < y_m$
\item $\exists$ an EMM $\mathbb{Q} \approx \mathbb{P}$ s.t. $\frac{\tilde S^1}{\tilde S^0}$ is a $\mathbb{Q}$-martingale.
\item The market $S$ is (NA).
\end{enumerate}

\paragraph{Completeness}

The multinomial model is
\begin{itemize}
\item \emph{complete} whenever $m \leq 2$ \\
(i.e. there are \textit{no} nodes that allow for more than two possible stock price evolutions)
\item \emph{incomplete} whenever $m > 2$ \\
(i.e. there is \textit{at least one} node that allows for more than two possible stock price evolutions)
\end{itemize}

\paragraph{Inequality of the payoffs of Asian and European call options}

\begin{itemize}
\item Consider a European call option $C_k^E = (\tilde S_k^1)^+$ and an Asian call option with
\begin{align*}
C_k^A &= \left( \frac{1}{k} \sum_{j=1}^k \tilde S_j^1 - K \right)^+
\end{align*}
\item Then it holds for the $\mathbb{Q}$-expectation (risk-neutral) of the two payoffs that:
\begin{align*}
\mathbb{E}_\mathbb{Q}\left[ \frac{C_k^A}{\tilde S_k^0} \right] &\leq \mathbb{E}_\mathbb{Q} \left[ \frac{C_k^E}{\tilde S_k^0} \right]
\end{align*}
\item \textit{Interpretation:} Since the volatility of an Asian style contingent claim is lower than the one of a European style contingent claim, the Asian option bears lower risk and thus yields also lower profit.
\end{itemize}

\paragraph{American options}

\begin{itemize}
\item Consider an American option with maturity $T$ and nonnegative adapted payoff process $U = (U_k)_{k=0,\ldots,T}$.
\item Then the arbitrage-free price process $\bar V = (\bar V_k)_{k=0,\ldots,T}$ w.r.t. $\mathbb{Q}$ can be expressed as a backward recursive scheme such as:
\begin{align*}
\bar V_T &= U_T \\
\bar V_k &= \max(U_k,\mathbb{E}_\mathbb{Q}[\bar V_{k+1}|\mathcal{F}_k]), \qquad \text{for } k=0,\ldots, T-1
\end{align*}
\end{itemize}

\columnbreak

\section{Martingales}

\subsection{Martingales}

\paragraph{Martingales}

\begin{itemize}
\item Let $(\Omega, \mathcal{F}, \mathbb{P})$ be a probability space with a filtration $\mathbb{F} = (\mathcal{F}_k)_{k=0,1, \ldots, T}$.
\item A (real-valued) stochastic process $X = (X_k)_{k=0,1, \ldots, T}$ is called a \emph{martingale} (w.r.t. $\mathbb{P}$ and $\mathbb{F}$) if:
\begin{enumerate}
\item $X$ is adapted to $\mathbb{F}$.
\item $X$ is $\mathbb{P}$-integrable in the sense that $X_k \in \mathcal{L}^1(\mathbb{P})$ for each $k$, i.e.
\begin{align*}
\mathbb{E}_\mathbb{P}[|X|] < \infty
\end{align*}
\item $X$ satisfies the martingale property:
\begin{align*}
\mathbb{E}_\mathbb{P} \left[ X_l | \mathcal{F}_k \right] = X_k \qquad \mathbb{P} \text{-a.s. for } k \leq l
\end{align*}
\end{enumerate}
\item \textit{Interpretation:} This means that the best prediction for the later value $X_l$ given the information $\mathcal{F}_k$ is just the current value $X_k$. Hence the changes in a martingale cannot be predicted. \\
In other words, a martingale describes a fair game in the sense that one cannot predict where it goes next.
\item A \emph{supermartingale} is defined the same but with the property
\begin{align*}
\mathbb{E}_\mathbb{P} [ X_l | \mathcal{F}_l] \leq X_k
\end{align*}
\item A \emph{submartingale} is defined the same but with the property
\begin{align*}
\mathbb{E}_\mathbb{P} [ X_l | \mathcal{F}_l] \geq X_k
\end{align*}
\end{itemize}

\paragraph{Equivalent Martingale Measure (EMM)}

\begin{itemize}
\item $\mathbb{Q}$ is an EMM for the stochastic process $(S_k)_{k \geq 0}$ iff $S$ is a martingale under $\mathbb{Q}$ and if $\mathbb{Q}$ is equivalent to $\mathbb{P}$.
\item We denote by $\mathbb{P}_e(S)$ the set of all EMMs for $S$.
\item The probability measure $\mathbb{Q}$ is equivalent to $\mathbb{P}$ ($\mathbb{Q} \approx \mathbb{P}$) iff:
\begin{enumerate}
\item $\mathbb{Q}[A] > 0 \iff \mathbb{P}[A] > 0$
\item $\mathbb{Q}[\Omega] = 1$
\end{enumerate}
\item A stochastic process $(S_k)_{k \geq 0}$ is a $\mathbb{Q}$-martingale iff:
\begin{enumerate}
\item $S_k$ is adapted to the considered filtration.
\item $S_k$ is integrable: $\mathbb{E}_\mathbb{Q}[|S_k|] < \infty$.
\item Martingale property: $\mathbb{E}_\mathbb{Q}[S_{k+1} | \mathcal{F}_k] = S_k$.
\end{enumerate}
\end{itemize}

\paragraph{Local martingale}

\begin{itemize}
\item An adapted process $X = (X_k)_{k=0,1, \ldots, T}$ null at 0 (i.e. with $X_0 = 0$) is called a local martingale (w.r.t. $\mathbb{P}$ and $\mathbb{F}$) if there exists a sequence of stopping times $(\tau_n)_{n \in \mathbb{N}}$ increasing to $T$ s.t. for each $n \in \mathbb{N}$, the stopped process $X^{\tau_n} = (X_{k \wedge \tau_n})_{k=0,1, \ldots, T}$ is a $(\mathbb{P}, \mathbb{F})$-martingale.
\item We then call $(\tau_n)_{n \in \mathbb{N}}$ a \emph{localising sequence}.
\item For any martingale $X$ and any stopping time $\tau$, the stopped process $X^\tau$ is again a martingale. \\
In particular, $\mathbb{E}_\mathbb{P}[X_{k \wedge \tau}] = \mathbb{E}_\mathbb{P}[X_0]$, $\forall k$.
\end{itemize}

\subsection{Stopping times}

\paragraph{Stopping time}

The stochastic process $S^\tau = (S_k^\tau)_{k=0,1, \ldots, T}$ defined by
\begin{align*}
S_k^\tau(\omega) := S_{k \wedge \tau}(\omega) := S_{k \wedge \tau(\omega)}(\omega)
\end{align*}
is called the process $S$ stopped at $\tau$. It clearly behaves like $S$ up to time $\tau$ and remains constant after time $\tau$.

\paragraph{Stopping theorem}

\begin{itemize}
\item Suppose that $M = (M_t)_{t \geq 0}$ is a $(\mathbb{P}, \mathbb{F})$-martingale with RC trajectories, and $\sigma, \tau$ are $\mathbb{F}$-stopping times with $\sigma \leq \tau$.
\item If either $\tau$ is bounded by some $T \in (0, \infty)$  or $M$ is uniformly integrable, then $M_\tau$, $M_\sigma$ are both in $L^1(\mathbb{P})$ and
\begin{align*}
\mathbb{E}[M_\tau | \mathcal{F}_\sigma] = M_\sigma \qquad \mathbb{P} \text{-a.s.}
\end{align*}
\end{itemize}

\paragraph{Cases of stopping times}

\begin{itemize}
\item Define the stopping time $\tau_a$ for $a \in \mathbb{R}, a > 0$ as:
\begin{align*}
\tau_a &:= \inf \lbrace t \geq 0 | W_t > a \rbrace
\end{align*}
Then it holds that:
\begin{itemize}
\item $\tau_{a_1} \leq \tau_{a_2}, \mathbb{P}$-a.s. for $a_1 < a_2$.
\item $\mathbb{P}[\tau_a < \infty] = 1$.
\item $W_{\tau_a} = a, \mathbb{P}$-a.s.
\item $\mathbb{E} \left[ \left. W_{\tau_{a_2}} \right| \mathcal{F}_{\tau_{a_1}} \right] \neq W_{\tau_{a_1}}, \mathbb{P}$-a.s. \\
i.e. the stopping theorem fails for $\tau = \tau_{a_2}$ and $\sigma = \tau_{a_1}$.
\item $\lim_{n \to \infty} \tau_n(\omega) = \infty, \mathbb{P}$-a.s.
\end{itemize}
\item Define the stopping time $\rho_a$ for $a \in \mathbb{R}, a > 0$ as:
\begin{align*}
\rho_a &:= \sup \lbrace t \geq 0 | W_t > a \rbrace
\end{align*}
Then it follows that $\rho_a = \infty$ with probability 1 under $\mathbb{P}$.
\end{itemize}

\subsection{Density processes/Girsanov's theorem}

\paragraph{Density in discrete time}

\begin{itemize}
\item Assume $(\Omega, \mathcal{F})$ and a filtration $\mathbb{F} = (\mathcal{F}_k)_{k=0,1,\ldots,T}$ in finite discrete time. \\
On $(\Omega, \mathcal{F})$, we have two probability measures $\mathbb{Q}$ and $\mathbb{P}$, and we assume $\mathbb{Q} \approx \mathbb{P}$.
\item \emph{Radon-Nykodin theorem:} There exists a density
\begin{align*}
\frac{d\mathbb{Q}}{d\mathbb{P}} &:= \mathcal{D}
\end{align*}
This is a RV $\mathcal{D} > 0$, $\mathbb{P}$-a.s. s.t. for all $A \in \mathcal{F}_k$ and for all RVs $Y \geq 0$ it holds that:
\begin{align*}
\mathbb{Q}[A] = \mathbb{E}_\mathbb{P}[\mathcal{D} \mathbb{I}_A], \qquad \mathbb{E}_\mathbb{Q}[Y] = \mathbb{E}_\mathbb{P}[Y \mathcal{D}].
\end{align*}
\item This can also be written as
\begin{align*}
\int_\Omega Y d\mathbb{Q} = \int_\Omega Y \mathcal{D} d\mathbb{P}
\end{align*}
This formula tells us how to compute $\mathbb{Q}$-expectations in terms of $\mathbb{P}$-expectations and vice-versa.
\end{itemize}

\paragraph{Density process in discrete time}

\begin{itemize}
\item Assume the same setting as before.
\item \emph{Radon-Nykodin theorem:} The density process $Z$ of $\mathbb{Q}$ w.r.t. $\mathbb{P}$, or also called the $\mathbb{P}$-martingale $Z$, is defined as
\begin{align*}
Z_k :&= \mathbb{E}_\mathbb{P}[\mathcal{D}| \mathcal{F}_k] = \mathbb{E}_\mathbb{P} \left[ \left. \frac{d\mathbb{Q}}{d\mathbb{P}} \right\vert \mathcal{F}_k \right] \qquad \text{for } k=0,1, \ldots, T
\end{align*}
\item Then for every $\mathcal{F}_k$-measurable RV $Y \geq 0$ or $Y \in \mathcal{L}^1(\mathbb{Q})$, it holds that
\begin{align*}
\mathbb{E}_\mathbb{Q}[Y | \mathcal{F}_k] = \mathbb{E}_\mathbb{P}[Y Z_k | \mathcal{F}_k]
\end{align*}
and for every $k \in \lbrace 0,1,\ldots, T \rbrace$ and any $A \in \mathcal{F}_k$, it holds that
\begin{align*}
\mathbb{Q}[A] &= \mathbb{E}_\mathbb{P}[Z_k \mathbb{I}_A]
\end{align*}
\item Properties:
\begin{itemize}
\item $Z_k$ is a RV and $Z_k > 0$, $\mathbb{P}$-a.s.
\item A process $N = (N_k)_{k=0,1, \ldots, T}$ which is adapted in $\mathbb{F}$ is a $\mathbb{Q}$-martingale iff the product $Z N$ is a $\mathbb{P}$-martingale. \\
(This tells us how martingale properties under $\mathbb{P}$ and $\mathbb{Q}$ are related to each other.)
\end{itemize}
\item \emph{Bayes formula:} \\
If $j \leq k$ and $U_k$ is $\mathcal{F}_k$-measurable and either $\geq 0$ or in $\mathcal{L}^1(\mathbb{Q})$, then
\begin{align*}
\mathbb{E}_\mathbb{Q} [U_k | \mathcal{F}_j] = \frac{1}{Z_j} \mathbb{E}_\mathbb{P}[Z_k U_k | \mathcal{F}_j] \qquad \mathbb{Q} \text{-a.s.}
\end{align*}
This tells us how conditional expectations under $\mathbb{Q}$ and $\mathbb{P}$ are related to each other.
\end{itemize}

\paragraph{Density process in continuous time}

\begin{itemize}
\item Suppose we have $\mathbb{P}$ and a filtration $\mathbb{F} = (\mathcal{F}_t)_{t \geq 0}$. \\
Fix $T \in (0,\infty)$ and assume only that $\mathbb{Q} \approx \mathbb{P}$ on $\mathcal{F}_T$.
\item Then the density process $Z$ of $\mathbb{Q}$ w.r.t. $\mathbb{P}$ on $[0,T]$ is defined as
\begin{align*}
Z_t :&= \mathbb{E}_\mathbb{P} \left[ \left. \frac{d\mathbb{Q}|_{\mathcal{F}_T}}{d\mathbb{P}|_{\mathcal{F}_T}} \right\vert \mathcal{F}_t \right] \qquad \text{for } 0 \leq t \leq T
\end{align*}
\item \emph{Bayes formula:} \\
For $s \leq t \leq T$ and every $U_t$ which is $\mathcal{F}_t$-measurable and either $\geq 0$ or in $L^1(\mathbb{Q})$, it holds that
\begin{align*}
\mathbb{E}_\mathbb{Q} [U_t | \mathcal{F}_s] = \frac{1}{Z_s} \mathbb{E}_\mathbb{P}[Z_t U_t | \mathcal{F}_s] \qquad \mathbb{Q} \text{-a.s.}
\end{align*}
\item It also holds that an adapted process $Y = (Y_t)_{0 \leq t \leq T}$ is a (local) $\mathbb{Q}$-martingale iff the product $Z Y$ is a (local) $\mathbb{P}$-martingale.
\end{itemize}

\paragraph{Girsanov's theorem}

\begin{itemize}
\item Suppose that $\mathbb{Q} \underset{\text{loc}}{\approx} \mathbb{P}$ with density process $Z$.
\item If $M$ is a local $\mathbb{P}$-martingale null at 0, then
\begin{align*}
\tilde M := M - \int \frac{1}{Z} d[Z,M]
\end{align*}
is a local $\mathbb{Q}$-martingale null at 0.
\item In particular, every $\mathbb{P}$-semimartingale is also a $\mathbb{Q}$-semimartingale (and vice-versa, by symmetry).
\end{itemize}

\paragraph{Girsanov (continuous version)}

\begin{itemize}
\item Suppose that $\mathbb{Q} \underset{\text{loc}}{\approx} \mathbb{P}$ with continuous density process $Z$. \\
Write $Z = Z_0 \mathcal{E}(L)$.
\item If $M$ is a local $\mathbb{P}$-martingale null at 0, then
\begin{align*}
\tilde M := M - [L,M] = M - \langle L,M \rangle
\end{align*}
is a local $\mathbb{Q}$-martingale null at 0.
\item Moreover, if $W$ is a $\mathbb{P}$-BM, then $\tilde W$ is a $\mathbb{Q}$-BM.
\item In particular, if $L = \int \nu dW$ for some $\nu \in \mathcal{L}_\text{loc}^2(W)$, then $\tilde W = W - \langle \int \nu dW, W \rangle = W - \int \nu_s ds$ so that the $\mathbb{P}$-BM $W = \tilde W + \int \nu_s ds$ becomes under $\mathbb{Q}$ a BM with (instantaneous) drift $\nu$.
\end{itemize}

\columnbreak

\section{Stochastic Integration and Calculus}

\subsection{Brownian motion \& Poisson processes}

\paragraph{Brownian motion}

\begin{itemize}
\item A Brownian motion w.r.t. $\mathbb{P}$ and a filtration $\mathbb{F} = (\mathcal{F}_t)_{t \geq 0}$ is a real-valued stochastic process $W = (W_t)_{t \geq 0}$ which satisfies the following properties:
\begin{description}[style=multiline,leftmargin=1cm,font=\normalfont]
\item[(BM0)] \textit{null at zero} \\
$W$ is adapted to $\mathbb{F}$ and null at 0 (i.e. $W_0 \equiv 0$, $\mathbb{P}$-a.s.).
\item[(BM1)] \textit{independent and stationary increments} \\
For $s \leq t$, the increment $W_t - W_s$ is independent (under $\mathbb{P}$) of $\mathcal{F}_s$ and satisfies under $\mathbb{P}$: $W_t - W_s \sim \mathcal{N}(0, t-s))$.
\item[(BM2)] \textit{continuous sample paths} \\
$W$ has continuous trajectories, i.e. for $\mathbb{P}$-a.a. $\omega \in \Omega$, the function $t \rightarrow W_t(\omega)$ on $[0, \infty)$ is continuous.
\end{description}
\end{itemize}
Remarks:
\begin{itemize}
\item Brownian motion in $\mathbb{R}^m$ is simply an adapted $\mathbb{R}^m$-valued stochastic process null at 0 and with the increment $W_t - W_s$ having the normal distribution $\mathcal{N}(0,(t-s) \mathbb{I}_{m \times m})$, where $\mathbb{I}_{m \times m}$ denotes the identity matrix.
\item For $\mathbb{P}$-a.a. $\omega \in \Omega$, the function $t \mapsto W_t(\omega)$ from $[0, \infty)$ to $\mathbb{R}$ is continuous, but \textit{nowhere differentiable}.
\end{itemize}

\paragraph{Transformations of BM}

The following stochastic processes are BMs:

\begin{itemize}
\item $W^1 := - W$
\item Restarting at a fixed time $T$:
\begin{align*}
W_t^2 := W_{T+t} -W_T
\end{align*}
for $t \geq 0$ and for any $T \in (0, \infty)$.
\item Rescaling in space and time:
\begin{align*}
W_t^3 := c W_{\frac{t}{c^2}}
\end{align*}
for $t \geq 0$ and for any $c \in \mathbb{R}$, $c \neq 0$.
\item Time-reversal on $[0,T]$:
\begin{align*}
W_t^4 := W_{T-t} - W_T
\end{align*}
for $0 \leq t \leq T$ and for any $T \in (0, \infty)$.
\item Inversion of small and large times:
\begin{align*}
W_t^5 :=
\begin{cases}
t W_{\frac{1}{t}} & \text{for } t > 0 \\
0 & \text{for } t = 0
\end{cases}
\end{align*}
for $t \geq 0$.
\item $W_t^6 := (W_t)^2 - t = 2 \int_0^t W_s dW_s, \qquad t \geq 0$
\item $W_t^7 := \exp \left( \alpha W_t - \frac{1}{2} \alpha^2 t \right)$ for $t \geq 0$ and for any $\alpha \in \mathbb{R}$.
\end{itemize}

\paragraph{Laws on BM}

\begin{itemize}
\item Law of large numbers:
\begin{align*}
\lim_{t \rightarrow \infty} \frac{W_t}{t} = 0, \qquad \mathbb{P}\text{-a.s.}
\end{align*}
i.e. BM grows asymptotically less than linearly (as $t \rightarrow \infty$).
\item Global law of the iterated logarithm (LIL): \\
With $\psi_\text{glob}(t) := \sqrt{2 t \log(\log t))}$, it holds for $t \geq 0$ that:
\begin{align*}
\lim_{t \rightarrow \infty} \sup \frac{W_t}{\psi_\text{glob}(t)} = +1 \\
\lim_{t \rightarrow \infty} \inf \frac{W_t}{\psi_\text{glob}(t)} = -1
\end{align*}
i.e. for $\mathbb{P}$-a.a. $\omega$, the function $t \rightarrow W_t(\omega)$ for $t \rightarrow \infty$ oscillates precisely between $t \rightarrow \pm \psi_\text{glob}(t)$.
\item Local law of the iterated logarithm (LIL): \\
With $\psi_\text{loc}(h) := \sqrt{2 h \log(\log \frac{1}{h})}$, it holds for $t \geq 0$ that:
\begin{align*}
\lim\limits_{h \searrow 0} \sup \frac{W_{t+h} - W_t}{\psi_\text{loc}(h)} = +1 \\
\lim\limits_{h \searrow 0} \inf \frac{W_{t+h} - W_t}{\psi_\text{loc}(h)} = -1
\end{align*}
i.e. for $\mathbb{P}$-a.a. $\omega$, to the right of $t$, the trajectory $u \rightarrow W_u(\omega)$ around the level $W_t(\omega)$ oscillates precisely between $h \rightarrow \pm \psi_\text{loc}(h)$.
\end{itemize}

\paragraph{Poisson processes}

\begin{itemize}
\item A Poisson process $N = (N_t)_{t \geq 0}$ with parameter $\lambda \in \mathbb{R}, \lambda > 0$ and w.r.t. $(\mathbb{P},\mathbb{F})$ is a real-valued stochastic process satisfying the following properties:
\begin{description}[style=multiline,leftmargin=1cm,font=\normalfont]
\item[(PP0)] \textit{null at zero} \\
$N$ is adapted to $\mathbb{F}$ and null at 0 (i.e. $N_0 \equiv 0, \mathbb{P}$-a.s.).
\item[(PP1)] \textit{independent and stationary increments} \\
For $0 \leq s < t$, the increment $N_t - N_s$ is independent (under $\mathbb{P}$) of $\mathcal{F}_s$ and follows (under $\mathbb{P}$) the Poisson distribution with parameter $\lambda (t-s)$, i.e. $N_t - N_s \sim \Poi(\lambda(t-s))$, i.e
\begin{align*}
\mathbb{P}[N_t=k] &= \frac{\lambda^k (t-s)^k}{k!} e^{-\lambda (t-s)}
\end{align*}
\item[(PP2)] \textit{counting process} \\
$N$ is a counting process with jumps of size 1, i.e. for $\mathbb{P}$-a.a. $\omega$, the function $t \mapsto N_t(\omega)$ is RCLL, piecewise constant and $\mathbb{N}_0$-valued, and increases by jumps of size 1.
\end{description}
\item Important properties of Poisson processes: if $X \sim \Poi(\lambda)$, then:
\begin{align*}
\mathbb{E}[X] = \lambda, \qquad \Var[X] = \lambda
\end{align*}
The quadratic variation of a Poisson process equals itself, i.e.
\begin{align*}
[N]_t &= N_t
\end{align*}
\item Examples of Poisson processes: the following Poisson processes are $(\mathbb{P},\mathbb{F})$-martingales:
\begin{itemize}
\item \emph{Compensated Poisson process:}
\begin{align*}
\tilde N_t = N_t - \lambda t, \qquad t \geq 0
\end{align*}
\item \emph{Geometric Poisson process:}
\begin{align*}
S_t &= \exp \left( N_t \log(1+\sigma) - \lambda \sigma t \right), \qquad t \geq 0
\end{align*}
where $\sigma \in \mathbb{R}, \sigma > -1$.
\item Two cases of squared compensated Poisson processes:
\begin{align*}
\left( \tilde N_t \right)^2 - N_t, \qquad \left( \tilde N_t \right)^2 - \lambda t, \qquad t \geq 0
\end{align*}
It follows that $[\tilde N]_t = N_t$.
\end{itemize}
\end{itemize}

\subsection{Stochastic integration}

\paragraph{Optional quadratic variation/square bracket process}

\begin{itemize}
\item For any local martingale $M = (M_t)_{t \geq 0}$ null at 0, there exists a unique adapted increasing RCLL process $[M] = ([M]_t)_{t \geq 0}$ and having the property that $M^2 - [M]$ is also a local martingale.
\item This process can be obtained as the quadratic variation of $M$ in the following sense. \\
There exists a sequence $(\Pi_n)_{n \in \mathbb{N}}$ of partitions $[0,)$ with $|\pi| \rightarrow 0$ as $n \rightarrow \infty$ s.t.
\begin{align*}
& \mathbb{P} \left[ [M]_t(\omega) = \lim\limits_{n \rightarrow \infty} \sum_{t_i \in \Pi_n} \left( M_{t_i \wedge t}(\omega) - M_{t_{i-1}\wedge t}(\omega) \right)^2 \forall t \geq 0 \right] \\
& \qquad = 1
\end{align*}
We call $[M]$ the optional quadratic variation or square bracket process of $M$.
\item If $M$ satisfies $\sup_{0 \leq s \leq t} |M_s| \in \mathcal{L}^2$ for each $t \geq 0$ (and hence is in particular a martingale), then $[M]$ is integrable (i.e. $[M]_t \in \mathcal{L}^1$ for every $t \geq 0$) and $M^2 - [M]$ is a martingale.
\end{itemize}

\paragraph{(Optional) covariation process}

\begin{itemize}
\item For two local martingales $M, N$ null at 0, we define the (optional) covariation process $[M,N]$ by polarisation, i.e.
\begin{align*}
[M,N] := \frac{1}{4} \left( [M+N] - [M-N] \right)
\end{align*}
\item The operation $[\cdot,\cdot]$ is bilinear.
\end{itemize}

\paragraph{Set of all bounded elementary processes}

\begin{itemize}
\item We denote by $b \mathcal{E}$ the set of all bounded elementary process of the form
\begin{align*}
H = \sum_{i=0}^{n-1} h_i \mathbb{I}_{(t_i,t_{i+1}]}
\end{align*}
with $n \in \mathbb{N}$, $0 \leq t_0 < t_1 < \ldots < t_n < \infty$ and each $h_i$ a bounded (real-valued) $\mathcal{F}_{t_i}$-measurable RV.
\end{itemize}

\paragraph{Stochastic integral}

\begin{itemize}
\item For any stochastic process $X = (X_t)_{t \geq 0}$, the stochastic integral $\int H dX$ of $H \in b \mathcal{E}$ is defined as
\begin{align*}
\int_0^t H_s dX_s := H \cdot X_t := \sum_{i=0}^{n-1} h_i \left( X_{t_{i+1} \wedge t} - X_{t_i \wedge t} \right) \quad \text{for } t \geq 0
\end{align*}
\item If $X$ and $H$ are both $\mathbb{R}^d$-valued, the integral is still real-valued, and we simply replace products by scalar products everywhere.
\item The following fundamental identity applies to the quadratic variation of stochastic integrals:
\begin{align*}
\left[ \int_0^t H_s dX_s \right] &= \int_0^t H_s^2 d[X]_s
\end{align*}
\end{itemize}

\paragraph{Isometry property}

\begin{itemize}
\item Suppose $M$ is a square-integrable martingale (i.e. $M_t \in \mathcal{L}^2$ for all $t \geq 0$).
\item For every $H \in b \mathcal{E}$, the stochastic integral process $H \cdot M = \int H dM$ is then also a square-integrable martingale, and we have the isometry property
\begin{align*}
\mathbb{E} \left[ (H \cdot M_\infty)^2 \right] = \mathbb{E} \left[ \int_0^\infty H_s^2 d[M]_s \right]
\end{align*}
\end{itemize}

\paragraph{Properties}

\begin{itemize}
\item (Local) Martingale properties
\item Linearity
\item Associativity
\item Behaviour under stopping
\item Quadratic variation and covaration
\item Jumps
\end{itemize}

\subsection{Stochastic calculus}

\paragraph{Itô's formula I}

\begin{itemize}
\item Suppose $X = (X_t)_{t \geq 0}$ is a continuous real-valued semimartingale and $f : \mathbb{R} \rightarrow \mathbb{R}$ is in $C^2$ (i.e. $f$ is twice continuously differentiable).
\item Then $f(X) = (f(X_t))_{t \geq 0}$ is again a continuous (real-valued) semimartingale, and we explicitly have $\mathbb{P}$-a.s. $\forall t \geq 0$:
\begin{align*}
f(X_t) &= f(X_0) + \int_0^t f'(X_s) dX_s + \frac{1}{2} \int_0^t f''(X_s) d \langle X \rangle_s \\
df(X_t) &= f'(X_t) dX_t + \frac{1}{2} f''(X_t) d \langle X \rangle_t
\end{align*}
\end{itemize}
Remarks:
\begin{itemize}
\item The $dX$-integral is a stochastic integral. It is well-defined since $X$ is a semimartingale and $f'(X)$ is adapted and continuous, hence predictable and locally bounded. \\
The $d\langle X \rangle$-integral is a classical Lebesgue-Stieltjes integral since $\langle X \rangle$ has increasing trajectories. It is also well-defined since $f''(X)$ is also predictable and locally bounded.
\item In comparison to the classical chain rule, the $d\langle X \rangle$-integral is an extra second-order term coming from the quadratic variation of $X$. Hence Itô's formula can be viewed as an extension of the chain rule.
\item The important message of this formula is that when one is dealing with stochastic models, a simple linear approximation is not good enough, since one also has to account for the second-order behaviour of $X$.
\item $\langle X \rangle_t = \langle M \rangle_t$
\end{itemize}

\paragraph{Itô's formula II}

\begin{itemize}
\item Suppose $X = (X_t)_{t \geq 0}$ is a general $\mathbb{R}^d$-valued semimartingale and $f : \mathbb{R}^d \rightarrow \mathbb{R}$ is in $C^2$.
\item Then $f(X) = (f(X_t))_{t \geq 0}$ is again a (real-valued) semimartingale, $\mathbb{P}$-a.s. and $\forall t \geq 0$.
\item If $X$ has continuous trajectories:
\begin{align*}
f(X_t) &= f(X_0) + \sum_{i=1}^d \int_0^t \frac{\partial f}{\partial x^i}(X_s) dX_s^i \\
& \qquad + \frac{1}{2} \sum_{i,j=1}^d \int_0^t \frac{\partial^2 f}{\partial x^i \partial x^j}(X_s) d \langle X^i, X^j \rangle_s
\end{align*}
\item If a stochastic process $X_t = f(t,W_t)$ with $f: \mathbb{R}^2 \to \mathbb{R}$ in $C^{1,2}$ (i.e. once continuously differentiable in time $t$ and twice continuously differentiable in $W_t$), then:
\begin{empheq}[box=\widefbox]{align*}
X_t &= X_0 + \underbrace{\int_0^t \frac{\partial f}{\partial w}(W_s,s) dW_s}\limits_{\text{local } (\mathbb{P},\mathbb{F}) \text{ martingale}} \\
& \qquad + \int_0^t \left( \frac{\partial f}{\partial t}(W_s,s) + \frac{1}{2} \frac{\partial^2 f}{\partial w^2}(W_s,s) \right) ds
\end{empheq}
Note that $X$ is a (continuous) local $(\mathbb{P},\mathbb{F})$-martingale iff
\begin{align*}
\int_0^t \left( \frac{\partial f}{\partial t}(W_s,s) + \frac{1}{2} \frac{\partial^2 f}{\partial w^2}(W_s,s) \right) ds = 0, \qquad \forall t \geq 0
\end{align*}
\item \emph{Itô's formula with jumps} \\
If $d=1$, $X$ real-valued but not necessarily continuous, then it holds that:
\begin{align*}
f(X_t) &= f(X_0) + \int_0^t f'(X_{s^-}) dX_s + \frac{1}{2} \int_0^t f''(X_{s^-}) d [X]_s \\
& \qquad + \sum_{0 < s \leq t} \left( f(X_s) - f(X_{s^-}) \right. \\
& \qquad \qquad \qquad \left. - f'(X_{s^-}) \Delta X_s - \frac{1}{2} f''(X_{s^-}) (\Delta X_s)^2 \right)
\end{align*}
\item If $f: \mathbb{R} \to \mathbb{R}$ is in $C^2$, $\alpha, \beta \in \mathbb{R}$ and the semimartingale $X = (X_t)_{t \geq 0}$ is given by $X_t = \alpha t + \beta N_t$, then:
\begin{align*}
f(X_t) &= f(X_0) + \alpha \int_0^t f'(X_{s-}) ds + \sum_{0<s \leq t} \left( f(X_s) - f(X_{s-}) \right)
\end{align*}
\end{itemize}

\paragraph{Stochastic exponential}

\begin{itemize}
\item For a general real-valued semimartingale $X$ null at 0, the stochastic exponential of $X$ is defined as the unique solution $Z$ of the SDE
\begin{align*}
dZ_t &= Z_{t^-} dX_t, \qquad Z_0 = 1
\end{align*}
and it follows that the unique solution to this SDE is:
\begin{align*}
Z_t := \mathcal{E}(X) &= 1 + \int_0^t Z_{s^-} dX_s \qquad \forall t \geq 0
\end{align*}
\begin{empheq}[box=\widefbox]{align*}
\mathcal{E}(X)_t &= \exp \left( X_t - \frac{1}{2} [X]_t \right)
\end{empheq}
\item \emph{Yor's formula:}
\begin{align*}
\mathcal{E}(X) \mathcal{E}(Y) = \mathcal{E}(X+Y+[X,Y])
\end{align*}
\end{itemize}

\paragraph{Itô process}

\begin{itemize}
\item An Itô process is a stochastic process of the form
\begin{align*}
X_t &= X_0 + \int_0^t \mu_s ds + \int_0^t \sigma_s dW_s, \qquad \forall t \geq 0
\end{align*}
where $W$ is some Brownian motion and $\mu$ and $\sigma$ are predictable processes.
\item More generally, $X, \mu, W$ could be vector-valued and $\sigma$ could be matrix-valued.
\item For any $C^2$ function $f$, the process $f(X)$ is again an Itô process, and Itô's formula gives
\begin{align*}
f(X_t) &= f(X_0) + \int_0^t \left( f'(X_s) \mu_s + \frac{1}{2} f''(X_s) \sigma_s^2 \right) ds \\
& \qquad + \int_0^t f'(X_s) \sigma_s dW_s
\end{align*}
\end{itemize}

\paragraph{Itô's representation theorem}

\begin{itemize}
\item Suppose that $W = (W_t)_{t \geq 0}$ is a $\mathbb{R}^m$-valued BM.
\item Then every RV $H \in \mathcal{L}^1(\mathcal{F}_\infty^W,P)$ has a unique representation as
\begin{align*}
H = \mathbb{E}[H] + \int_0^\infty \psi_s dW_s, \qquad \text{a.s.}
\end{align*}
for an $\mathbb{R}^m$-valued integrand $\psi \in \mathcal{L}_\text{loc}^2(W)$.
\item $\psi$ has the additional property that $\int \psi dW$ is a $(P, \mathbb{F}^W)$-martingale on $[0,\infty]$ (and is thus uniformly integrable).
\end{itemize}

\paragraph{Itô product formula}

\begin{itemize}
\item Define the stochastic process $Z = X Y$, where $X$ and $Y$ are two continuous real-valued semimartingales.
\item Then $Z$ can be written as the sum of stochastic integrals:
\begin{align*}
Z_t - Z_0 &= \int_0^t Y_s dX_s + \int_0^t X_s dY_s + \int_0^t d[X,Y]_s
\end{align*}
\end{itemize}

\paragraph{General properties/results}

\begin{itemize}
\item Any continuous, adapted process $H$ is also predictable and locally bounded. \\
It furthermore holds for any predictable, locally bounded process $H$ that $H \in L_\text{loc}^2(W)$.
\item Let $f: \mathbb{R} \to \mathbb{R}$ be an arbitrary continuous convex function. Then the process $(f(W_t))_{t \geq 0}$ is integrable and is a $(\mathbb{P},\mathbb{F})$-submartingale.
\item Given a $(\mathbb{P},\mathbb{F})$-martingale $(M_t)_{t \geq 0}$ and a measurable function $g: \mathbb{R}_+ \to \mathbb{R}$, the process
\begin{align*}
(M_t + g(t))_{t \geq 0}
\end{align*}
is a:
\begin{itemize}
\item $(\mathbb{P},\mathbb{F})$-supermartingale iff $g$ is decreasing;
\item $(\mathbb{P},\mathbb{F})$-submartingale iff $g$ is increasing.
\end{itemize}
\item A continuous local martingale of finite variation is identically constant (and hence vanishes if it is null at 0).
\item For a function $f: \mathbb{R} \to \mathbb{R}$ in $C^1$, the stochastic integral $\int_0^\cdot f'(W_s) dW_s$ is a continuous local martingale. \\
Furthermore, for $f \in C^2$ it holds that $f(W)$ is a continuous local martingale iff $\int_0^\cdot f''(W_s) ds = 0$.
\item If a predictable process $H = (H_t)_{t \geq 0}$ satisfies
\begin{align*}
\mathbb{E} \left[ H_s^2 ds \right] < \infty, \qquad \forall T \geq 0
\end{align*}
then $\int H dW_s$ is a square-integrable martingale.
\item If $f: \mathbb{R} \to \mathbb{R}$ is bounded and continuous, then the stochastic integral $\int f(W) dW$ is a square-integrable martingale.
\item If a process $H = (H_t)_{t \geq 0}$ is predictable and the map $s \mapsto \mathbb{E}[H_s^2]$ is continuous, then the stochastic integral $\int H dW$ is a square-integrable martingale.
\item If $f: \mathbb{R} \to \mathbb{R}$ is polynomial, then the stochastic integral $\int f(W) dW$ is a square-integrable martingale.
\end{itemize}

\columnbreak

\section{Black-Scholes Formula}

\subsection{Black-Scholes (BS) model}

\paragraph{BS model (undiscounted, historical measure $\mathbb{P}$)}

\begin{align*}
\tilde S_t^0 &= e^{r t} &
\tilde S_t^1 &= \tilde S_0^1 \exp \left( \sigma W_t + \left( \mu - \frac{1}{2} \sigma^2 \right) t \right) \\
\frac{d\tilde S_t^0}{\tilde S_t^0} &= r dt & \frac{d\tilde S_t^1}{\tilde S_t^1} &= \mu dt + \sigma dW_t
\end{align*}

\paragraph{BS model (discounted, historical measure $\mathbb{P}$)}

\begin{align*}
 S_t^0 &= 1 &
 S_t^1 &=  S_0^1 \exp \left( \sigma W_t + \left( \mu - r - \frac{1}{2} \sigma^2 \right) t \right) \\
& & \frac{d  S_t^1}{S_t^1} &= (\mu - r) dt + \sigma dW_t
\end{align*}

\paragraph{BS model (discounted, risk-neutral measure $\mathbb{Q}$)}

\begin{align*}
dS_t^1 &= S_t^1 \sigma \left( dW_t + \frac{\mu - r}{\sigma} dt \right) = S_t^1 \sigma dW_t^\ast \\
S_t^1 &= S_0^1 + \int_0^t S_u^1 \sigma dW_u^\ast = S_0^1 \exp \left( \sigma W_t^\ast - \frac{1}{2} \sigma^2 t \right)
\end{align*}
where
\begin{align*}
W_t^\ast &:= W_t + \frac{\mu - r}{\sigma} t = W_t + \int_0^t \lambda ds
\end{align*}

\paragraph{Market price of risk}

\begin{itemize}
\item The market price of risk or infinitemsimal \emph{Sharpe ratio} of $S^1$ is defined as
\begin{align*}
\lambda^\ast = \frac{\mu - r}{\sigma}
\end{align*}
\end{itemize}

\subsection{Black-Scholes PDE}

\paragraph{Black-Scholes PDE}

\begin{align*}
0 &= \frac{\partial \tilde v}{\partial t} + r \tilde x \frac{\partial \tilde v}{\partial \tilde x} + \frac{1}{2} \sigma^2 \tilde x^2 \frac{\partial^2 \tilde v}{\partial \tilde x^2} - r \tilde v, \qquad \tilde v(T, \cdot) = \tilde h(\cdot)
\end{align*}

\subsection{Black-Scholes formula for option pricing}

\paragraph{Martingale pricing approach}

\begin{itemize}
\item The discounted arbitrage-free value at time $t$ of any discounted payoff $H \in L_+^1(\mathcal{F}_T,\mathbb{Q}^\ast)$, $H_T = H(\tilde S^0_T, \tilde S^1_T)$, is given by
\begin{empheq}[box=\widefbox]{align*}
V_t^\ast &= \mathbb{E}_\mathbb{Q} \left[ H | \mathcal{F}_t \right] := \vartheta(t,S^1_t)
\end{empheq}
\item Then the discounted payoff $H$ can be hedged via the replicating strategy $(V_0,\vartheta)$ s.t.
\begin{align*}
V_0 + \int_0^T \vartheta_u dS^1_u = H(\tilde S^0_T, \tilde S^1_T)
\end{align*}
Using Itô's representation theorem the replicating strategy can be expressed as
\begin{empheq}[box=\widefbox]{align*}
V_t^\ast &= \vartheta(t,S^1_t) = V_0 + \int_0^t \vartheta_s dS^1_s + \underbrace{\textit{cont. FV process}}\limits_\text{''usually'' vanishes} \\
V_0 &= \vartheta(0,S^1_0), \qquad \vartheta_t = \frac{\partial \vartheta}{\partial x}(t,S^1_t)
\end{empheq}
\end{itemize}

\paragraph{Black-Scholes formula for a European call option}

\begin{align*}
\tilde V_t^{\tilde H} &= \tilde v(t, \tilde S_t^1) = \tilde S_t^1 \Phi(d_1) - \tilde K e^{-r(T-t)} \Phi(d_2) \\
d_{1,2} &= \frac{\log \left( \frac{\tilde S_t^1}{\tilde K} \right) + \left( r \pm \frac{1}{2} \sigma^2 \right) (T-t)}{\sigma \sqrt{T-t}} \\
\Phi(y) &= Q^\ast [Y \leq y] = \int_{-\infty}^y \frac{1}{\sqrt{2 \pi}} e^{-\frac{1}{2}z^2} dz
\end{align*}
Remarks:
\begin{itemize}
\item $\Phi$ denotes the CDF of the standard normal distribution $\mathcal{N}(0,1)$.
\item Note that the drift $\mu$ of the stock does not appear here. This is analogous to the result that the probability $p$ of an up move in the CRR binomial model does not appear in the binomial option pricing formula.
\end{itemize}

\paragraph{Replicating strategy for a European call option}

\begin{align*}
\nu_t^H &= \frac{\partial \tilde \nu}{\partial \tilde x}(t, \tilde S_t^1) = \Phi(d_1)
\end{align*}

\paragraph{Greeks}

The derivatives of the option price w.r.t. the various parameters, i.e. the sensitivities of the option price w.r.t. to the parameters, are called Greeks.

\columnbreak

\section{Appendix}

\paragraph{Markov's inequality}

For $X$ any nonnegative integrable RV and $a \in \mathbb{R}, a > 0$:
\begin{align*}
\mathbb{P}[X \geq a] &\leq \frac{\mathbb{E}[X]}{a}
\end{align*}

\paragraph{Chebyshev's inequality}

For $X$ an integrable RV with finite expected value $\mu \in \mathbb{R}$ and finite non-zero variance $\sigma^2, \sigma \in \mathbb{R}$ and for any real number $k > 0$:
\begin{align*}
\mathbb{P}[|X - \mu| \geq k \sigma] \leq \frac{1}{k^2}
\end{align*}

\paragraph{Jensen's inequality}

For $X$ a RV and $f$ a convex function, it holds that
\begin{align*}
f(\mathbb{E}[X]) \leq \mathbb{E}[f(X)]
\end{align*}

\paragraph{Sets/families}

\begin{itemize}
\item $L_+^0$: familiy of all nonnegative RVs
\end{itemize}

\paragraph{Correlation and Independence}

\begin{itemize}
\item Let $X$ and $Y$ be two RVs.
\item Then $X,Y$ are \emph{uncorrelated} iff
\begin{align*}
\mathbb{E}[X \cdot Y] &= \mathbb{E}[X] \cdot \mathbb{E}[Y]
\end{align*}
\item Then $X,Y$ are \emph{independent} iff
\begin{align*}
\mathbb{P}[X=x,Y=y] &= \mathbb{P}[X=x] \cdot \mathbb{P}[Y=y]
\end{align*}
Note that independence of $X,Y$ implies that $X,Y$ are uncorrelated (but not vice-versa!).
\end{itemize}

\paragraph{Independence of equations}

\begin{itemize}
\item If there is e.g. a system of equations such as
\begin{align*}
\begin{cases}
a_1 x + b_1 y = c_1 \\
a_2 x + b_2 y = c_2 \\
a_3 x + b_3 y = c_3
\end{cases}
\end{align*}
then this system admits a solution iff
\begin{align*}
\det \left[
\begin{matrix}
a_1 & b_1 & c_1 \\
a_2 & b_2 & c_2 \\
a_3 & b_3 & c_3
\end{matrix}
\right] &= 0
\end{align*}
\end{itemize}

\paragraph{Fubini's lemma}

\begin{align*}
\mathbb{E} \left[ \int_0^T H_s^2 ds \right] &= \int_0^T \mathbb{E} \left[ H_s^2 \right] ds
\end{align*}

\columnbreak
\begin{center}
\textit{intentionally left blank}
\end{center}
\vfill

\section*{Abbreviations}

\begin{description}[style=multiline,leftmargin=1cm,font=\textbf]
\item[a.a.] almost all
\item[a.s.] almost surely
\item[BM] Brownian motion
\item[CDF] cumulative distribution function
\item[iff] if and only if
\item[IOT] in order to
\item[PDE] partial differential equation
\item[PDF] probability density function
\item[RCLL] right-continuous with left limits
\item[RV] random variable
\item[SDE] stochastic differential equation
\item[s.t.] such that
\item[w.r.t.] with respect to
\end{description}

\section*{Disclaimer}

\begin{itemize}
\item This summary is work in progress, i.e. neither completeness nor correctness of the content are guaranteed by the author.
\item This summary may be extended or modified at the discretion of the readers.
\item Source: Lecture Mathematical Foundations for Finance, autumn semester 2015/16, ETHZ (lecture notes, script and exercises). Copyright of the content is with the lecturers.
\item The layout of this summary is mainly based on the summaries of several courses of the BSc ETH ME from Jonas LIECHTI.
\end{itemize}

\end{multicols*}

\end{document}